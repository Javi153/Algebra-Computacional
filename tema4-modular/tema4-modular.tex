\documentclass[a4paper, 11pt]{article}

\usepackage[utf8]{inputenc}
\usepackage{amsmath,amssymb,amsthm}
\usepackage[margin=2cm]{geometry}
\usepackage{listings}
\usepackage{color}
\usepackage{mathtools}

\definecolor{deepblue}{rgb}{0,0,0.5}
\definecolor{deepred}{rgb}{0.6,0,0}
\definecolor{deepgreen}{rgb}{0,0.5,0}
\definecolor{lightgrey}{rgb}{0.95,0.95,0.95}

\renewcommand*{\lstlistingname}{Programa}

\lstset{
   language=Python,
   basicstyle=\ttfamily\small,
   keywordstyle=\color{deepblue}\bfseries\itshape,
   commentstyle=\color{deepgreen}\itshape,
   stringstyle=\color{deepred},
   backgroundcolor=\color{lightgrey},
   morekeywords={as,assert,nonlocal,with,yield},
   showstringspaces=false,
   numbers=left,
   rulecolor=\color{black},
   captionpos=b,
   frame=leftline,
   literate=
   {á}{{\'a}}1 {é}{{\'e}}1 {í}{{\'i}}1 {ó}{{\'o}}1 {ú}{{\'u}}1
   {Á}{{\'A}}1 {É}{{\'E}}1 {Í}{{\'I}}1 {Ó}{{\'O}}1 {Ú}{{\'U}}1
   {à}{{\`a}}1 {è}{{\`e}}1 {ì}{{\`i}}1 {ò}{{\`o}}1 {ù}{{\`u}}1
   {À}{{\`A}}1 {È}{{\`E}}1 {Ì}{{\`I}}1 {Ò}{{\`O}}1 {Ù}{{\`U}}1
   {ä}{{\"a}}1 {ë}{{\"e}}1 {ï}{{\"i}}1 {ö}{{\"o}}1 {ü}{{\"u}}1
   {Ä}{{\"A}}1 {Ë}{{\"E}}1 {Ï}{{\"I}}1 {Ö}{{\"O}}1 {Ü}{{\"U}}1
   {â}{{\^a}}1 {ê}{{\^e}}1 {î}{{\^i}}1 {ô}{{\^o}}1 {û}{{\^u}}1
   {Â}{{\^A}}1 {Ê}{{\^E}}1 {Î}{{\^I}}1 {Ô}{{\^O}}1 {Û}{{\^U}}1
   {ã}{{\~a}}1 {ẽ}{{\~e}}1 {ĩ}{{\~i}}1 {õ}{{\~o}}1 {ũ}{{\~u}}1
   {Ã}{{\~A}}1 {Ẽ}{{\~E}}1 {Ĩ}{{\~I}}1 {Õ}{{\~O}}1 {Ũ}{{\~U}}1
   {œ}{{\oe}}1 {Œ}{{\OE}}1 {æ}{{\ae}}1 {Æ}{{\AE}}1 {ß}{{\ss}}1
   {ű}{{\H{u}}}1 {Ű}{{\H{U}}}1 {ő}{{\H{o}}}1 {Ő}{{\H{O}}}1
   {ç}{{\c c}}1 {Ç}{{\c C}}1 {ø}{{\o}}1 {Ø}{{\O}}1 {å}{{\r a}}1 {Å}{{\r A}}1
   {€}{{\euro}}1 {£}{{\pounds}}1 {«}{{\guillemotleft}}1
   {»}{{\guillemotright}}1 {ñ}{{\~n}}1 {Ñ}{{\~N}}1 {¿}{{?`}}1 {¡}{{!`}}1 
}

\newcommand{\NN}{\mathbb{N}}
\newcommand{\ZZ}{\mathbb{Z}}
\newcommand{\QQ}{\mathbb{Q}}
\newcommand{\RR}{\mathbb{R}}
\newcommand{\CC}{\mathbb{C}}
\newcommand{\FF}{\mathbb{F}}
\newcommand{\UU}{\mathcal{U}}

\newcounter{numerodetema}
\newcommand\tema[1]{{\eject\stepcounter{numerodetema}\bf Álgebra Computacional
2023/24 \hfill Tema~\#\arabic{numerodetema}}\par\smallskip\hrule\bigskip\par
\begin{center}{\Large\bf #1}\end{center}}

\theoremstyle{plain}
\newtheorem{teor}{Teorema}[numerodetema]
\newtheorem{coro}[teor]{Corolario}
\newtheorem{lema}[teor]{Lema}
\newtheorem{prop}[teor]{Proposición}
\theoremstyle{definition}
\newtheorem{defi}[teor]{Definición}
\newtheorem{prob}{Problema}[numerodetema]

\newcommand\myatop[2]{\genfrac{}{}{0pt}{}{#1}{#2}}

\parindent=0cm

\begin{document}

\setcounter{numerodetema}{3}
\tema{Aritmética modular}

Hasta ahora hemos visto como se pueden implementar las operaciones aritméticas
para enteros representados en forma decimal. Python3 utiliza internamente
una representación en base $2^{64}$ y llama a la librería GMP para hacer las
operaciones aritméticas, que tiene implementados unos algoritmos algo mas
eficientes que los que hemos estudiado. La siguiente tabla muestra las
complejidades binarias (operaciones entre dígitos) de cada una de las
implementaciones en función de la cantidad de dígitos totales de ambos
operandos:

\bigskip

\begin{center}
\begin{tabular}{|c|c|c|c|}
\hline
Operación en $\ZZ$ & Algoritmos de la escuela & Clases anteriores & Python3/GMP \\
\hline
Suma, resta y comparación  & $O(n)$ & $O(n)$  & $O(n)$ \\
Multiplicación  & $O(n^2)$ & $O(n^{\log_23})$ & $O(n\log(n)\log\log(n))$ \\
División con resto & $O(n^2)$ & $O(M(n))$ & $O(M(n))$ \\
\hline
\end{tabular}
\end{center}

\bigskip

Otro algoritmo importante que vamos a utilizar es el que calcula el divisor
común mayor $g$ entre dos enteros $x,y$ dados y además obtiene $a,b$ tales
que $g=ax+by$. En el problema 3.7 vimos que eso puede hacerse en $O(n^2)$
operaciones binarias, donde $n$ es la cantidad total de dígitos de $x$ e $y$.
El algoritmo de Euclides tiene la misma complejidad binaria $O(n^2)$.

\bigskip

Sea $N\geq 1$ un entero. Decimos que $a\equiv b\pmod{N}$ si y solo si
$N\mid a-b$, es decir, si $a$ y $b$ tienen el mismo resto al dividirlos
por $N$. Se comprueba inmediatamente que la relación de congruencia
módulo~$N$ es reflexiva, simétrica y transitiva. Eso separa a los
enteros en clases disjuntas. Por ejemplo, si $N=2$, todos los números
pares están en una misma clase y todos los impares en otra. En álgebra,
se suele representar $[a]_N$ a la clase de equivalencia módulo $N$ que
contiene a $a$, es decir, \[[a]_N=\{b\in\ZZ\,:\,b\equiv a\pmod{N}\}.\]
Cuando no haya ambigüedad con el valor de $N$, se puede simplemente
indicar $[a]$, o incluso, $\overline{a}$.

\bigskip

La relación de congruencia módulo $N$ satisface
\[
a\equiv b\;\wedge c\equiv d\;\Longrightarrow\;
a+c\equiv b+d,\; a-c\equiv b-d,\; ac\equiv bd\pmod{N}
\]
por lo que el conjunto de clases de equivalencia forma un anillo (con la suma
y producto usual) que se denota $\ZZ/N\ZZ$. 

\bigskip

Los elementos del anillo $\ZZ/N\ZZ$ pueden \emph{representarse} unívocamente
con los enteros entre $0$ y $N-1$.
\[
\ZZ/N\ZZ=\big\{ [0], [1], [2],\ldots,[N-1] \big\}
\]
De aquí en adelante, utilizaremos esa representación. La operación
\texttt{a \% N} de Python3 permite reducir un entero \texttt{a} cualquiera
a su correspondiente representante en el rango $[0,N-1]$. En álgebra,
es común escribir esa operación como $a\bmod N$.

\bigskip

Si $n=size(N)\approx\log_{10}(N)$, entonces las sumas y restas en $\ZZ/N\ZZ$
pueden hacerse siempre con $O(n)$ operaciones binarias. Por ejemplo, una suma
en $\ZZ/N\ZZ$ requiere de una suma en $\ZZ$ de dos números de $n$ dígitos,
luego una comparación en $\ZZ$ con $N$ para saber si el resultado se salió del
rango $[0,N-1]$ y, en ese caso, una resta en $\ZZ$ con $N$. En el peor caso son
tres operaciones en $\ZZ$ de números de a lo sumo $n+1$ dígitos, lo que tiene
complejidad binaria~$O(n)$. De forma similar se puede ver que la resta en
$\ZZ/N\ZZ$ tiene también complejidad binaria~$O(n)$.

\bigskip

La multiplicación en $\ZZ/N\ZZ$ se reduce a una multiplicación de enteros
de $n$ dígitos, cuyo resultado tendrá $2n$ dígitos, seguida de una división
con resto por $N$. Esto tiene complejidad binaria $O(M(n))$, dependiendo de
la implementación elegida para la multiplicación de enteros, y suponiendo
que se use el método de Netwon para la división.

\bigskip

\lstinputlisting[language=Python,linerange={1-16},firstnumber=1,
caption={\texttt{modular.py} (lineas 1--16)},label={prog50}]
{modular.py}

\bigskip

Una de las operaciones mas importantes que vamos a necesitar es la
potenciación. Una forma simple de calcular $a^k$ para $a\in\ZZ/N\ZZ$ y
$k\geq 0$ es mediante $k$ multiplicaciones. Un bucle \texttt{for i in range(k):}
es suficiente, pero lamentablemente eso hace que la complejidad binaria del
método sea \textit{exponencial} en el tamaño de $k$. Una mejor idea es
utilizar el mismo truco que hicimos para el divisor común mayor, es decir,
si $k$ es par, calculamos primero $a^{k/2}$ y luego elevamos al cuadrado (una
sola multiplicación), y si $k$ es impar, hacemos $a^k=a\cdot a^{k-1}$ para
reducirnos al caso par. De ese modo, solamente haremos $O(\log k)$ multiplicaciones
en $\ZZ/N\ZZ$, lo que da una complejidad total $O(M(n)\log k)$.

\bigskip

\lstinputlisting[language=Python,linerange={18-27},firstnumber=18,
caption={\texttt{modular.py} (lineas 18--27)},label={prog51}]
{modular.py}

\begin{prob}
Demostrar que la sucesión de Fibonacci satistace
\[
   \left[\begin{array}{c} f_{n} \\ f_{n+1} \end{array}\right] =
   \left[\begin{array}{cc} 0 & 1 \\ 1 & 1 \end{array}\right]^n\cdot
   \left[\begin{array}{c} 0 \\ 1 \end{array}\right]
\]
y utilizar esa identidad para implementar un algoritmo que calcule
eficientemente $f_k\bmod N$ dados $k\geq 0$ y $N\geq 1$. ¿Cuál es
la complejidad binaria del algoritmo?
\end{prob}

\bigskip

Sean $N_1,N_2\geq 1$ tales que $N_1\mid N_2$. Esto implica que si
$a\equiv b\pmod{N_2}$ entonces también $a\equiv b\pmod{N_1}$, o en
términos mas algebraicos, que la aplicación
\[
\begin{aligned}
\ZZ/N_2\ZZ & \rightarrow\ZZ/N_1\ZZ \\
[a]_{N_2} &\mapsto [a]_{N_1}
\end{aligned}
\]
está bien definida y es un morfismo de anillos. Por ejemplo, conocer
la clase de un entero módulo $10$, permite facilmente determinar la
clase de equivalencia módulos $2$ (la paridad) y módulo $5$.

\begin{teor}[Teorema chino del resto]\label{teor_chino}
Sean $N,M\geq 1$ coprimos, es decir, $\gcd(N,M)=1$. Entonces la aplicación
\[
\begin{aligned}
\gamma:\ZZ/NM\ZZ & \rightarrow\ZZ/N\ZZ\oplus\ZZ/M\ZZ \\
[a]_{NM} & \mapsto ([a]_N,[a]_M)
\end{aligned}
\]
es un isomorfismo de anillos. En particular, para un
$N=p_1^{n_1}p_2^{n_2}\cdots p_k^{n_k}$ con $p_1,\ldots,p_k$ primos
distintos y $n_1,\ldots,n_k\geq 1$ se tiene que
$\ZZ/N\ZZ\simeq\ZZ/p_1^{n_1}\ZZ\oplus\cdots\oplus\ZZ/p_k^{n_k}\ZZ$.
\end{teor}
\begin{proof}
Ya hemos visto que $\gamma$ está bien definido y es un morfismo de anillos.
Para comprobar que $\gamma$ es un monomorfismo, basta ver que el núcleo de
$\gamma$ es trivial. Para eso, tomemos un elemento $[x]_{NM}\in\ZZ/NM\ZZ$ tal
que $\gamma([x]_{NM})=([0]_N,[0]_M)$. Eso significa que tenemos un entero $x$
que es divisible por~$N$ y por~$M$, y por lo tanto, también por $NM$, ya que
$N$ y $M$ no tienen factores en común. Eso prueba que $[x]_{NM}=[0]_{NM}$ en
$\ZZ/NM\ZZ$ como necesitabamos probar. La suryectividad sigue de que ambos
conjuntos tienen la misma cantidad de elementos y que la aplicación $\gamma$
es inyectiva.
\end{proof}

La demostración del teorema anterior es \textit{intencionalmente} no
constructiva, con el objetivo de mostrar un problema muy frecuente el álgebra
computacional. Hay una cantidad enorme de resultados de existencia y unicidad
en álgebra (biyecciones entre objetos) cuya demostración no da un procedimiento
explícito para calcular los objetos implicados.

\bigskip

Es claro que la aplicación $\gamma:\ZZ/NM\ZZ\to\ZZ/N\ZZ\oplus\ZZ/M\ZZ$ del
teorema~\ref{teor_chino} se puede implementar usando la operación $\%$ de
Python3:

\bigskip

\begin{lstlisting}
def gamma(x, N, M):
    return (x%N, x%M)
\end{lstlisting}

\bigskip

Sin embargo no queda nada claro como implementar la inversa
$\gamma^{-1}: \ZZ/N\ZZ\oplus\ZZ/M\ZZ\to\ZZ/NM\ZZ$. Si se lee cuidadosamente
la demostración, se puede ver que la existencia y buena definición de
$\gamma^{-1}$ se sigue de la inyectividad de $\gamma$ y que el dominio y
codominio de $\gamma$ tienen la misma cantidad de elementos. Es decir,
$\gamma^{-1}(y)$ existe y está bien definido porque $\gamma(x)$ recorre
todos los posibles elementos de $\ZZ/N\ZZ\oplus\ZZ/M\ZZ$ sin repetir cuando
$x$ recorre todos los elementos de  con $\ZZ/NM\ZZ$. Eso es el equivalente
a hacer una búsqueda exhaustiva.

\bigskip

\begin{lstlisting}
def gamma_inversa(y, N, M):
    for x in range(N*M):
        if gamma(x, N, M) == y:
            return x
\end{lstlisting}

\bigskip

Por supuesto, tal argumento es perfectamente válido para probar el 
teorema~\ref{teor_chino}, pero no es de esperarse que el método sea
el más eficiente.

\bigskip

Supongamos que $[x_1]_{NM}=\gamma^{-1}([1]_N,[0]_M)$ y
$[x_2]_{NM}=\gamma^{-1}([0]_N,[1]_M)$, es decir,
\begin{equation}\label{eq_interp}
\begin{aligned}
   x_1\equiv 1\pmod{N} \qquad &x_1\equiv 0\pmod{M} \\
   x_2\equiv 0\pmod{N} \qquad &x_2\equiv 1\pmod{M}
\end{aligned}
\end{equation}
para ciertos enteros $0\leq x_1,x_2<NM$. Entonces, se tiene que
$\gamma^{-1}([y_1]_N,[y_2]_M)=[x_1y_1+x_2y_2]_{NM}$.

\bigskip

El problema se reduce a entender como determinar $x_1$. Como $x_1\equiv0\pmod{M}$,
buscamos un múltiplo de $M$, es decir, $x_1=Mz_1$ para cierto $0\leq z_1<N$. La
otra ecuación nos dice que $Mz_1\equiv 1\pmod{N}$. Si aplicamos el algoritmo
del divisor común mayor extendido a $N$ y $M$ obtendríamos que $1=\gcd(N,M)=aN+bM$
para ciertos $a,b\in\ZZ$. Esa identidad nos dice que $bM\equiv1\pmod{N}$ y por
lo tanto podríamos tomar $z_1=b\bmod{N}$ como la solución de la ecuación que
nos falta.

\bigskip

\begin{lstlisting}
def gamma_inversa_eficiente(y, N, M):
    g, a, b = gcd_extendido_binario(N, M)  # obtenemos la tupla (g,a,b)
    x1 = M * (b % N)                       # x1=1 (mod N), x1=0 (mod M)
    x2 = N * (a % M)                       # x2=0 (mod N), x2=1 (mod M)
    x = (x1*y[0] + x2*y[1]) % (N*M)        # combinacion lineal
    return x
\end{lstlisting}

\bigskip

La hipótesis $\gcd(N,M)=1$ implica que en la linea 2 obtendremos $g=1$ y
que $1=aN+bM$ tal como necesitamos.

\begin{prob}\label{prob:cong1}
Sean $N_1,\ldots,N_k\geq 1$ coprimos y sea $N=\prod_{i=1}^kN_i$. Demostrar que
el sistema de congruencias simultáneas $x\equiv a_i\pmod{N_i}$ para
$a_1,\ldots,a_k$ dados es equivalente a una única congruencia
$x\equiv a\pmod{N}$ para algún $a$. Implementar este resultado en Python3 como
la función \texttt{reducir\_congruencias(l)} que toma una lista
\texttt{l = [(a1,N1),(a2,N2),\ldots,(ak,Nk)]} de pares ordenados (tuplas)
y devuelve el par \texttt{(a,N)}.
\end{prob}

\begin{prob}
Generalizar el resultado del problema~\ref{prob:cong1} al caso en que
$N_1,\ldots,N_k$ no sean necesariamente coprimos. Más concretamente,
probar que dados $a_1,\ldots,a_k\in\ZZ$, el sistema de congruencias
simultáneas $x\equiv a_i\pmod{N_i}$ tiene solución si y solo si
$a_i\equiv a_j\pmod{\gcd(N_i,N_j)}$ para todo $i,j$. En caso de
tener solución, esta es equivalente a una única congruencia de la
forma $x\equiv a\pmod{N}$, donde $N={\rm lcm}(N_1,N_2,\ldots,N_k)$.
\end{prob}

\bigskip

En la implementación de \texttt{gamma\_inversa\_eficiente(x,y,N)} hemos
usado dos trucos importantes que se repiten muchas veces en álgebra
computacional. El primero es la interpolación: basta con resolver las
ecuaciones~\eqref{eq_interp} y luego hacer combinaciones lineales de
esas soluciones. El otro truco es obtener inversas modulares utilizando
que la identidad de Bezout $gcd(x,y)=ax+by$ puede obtenerse algoritmicamente
de forma eficiente. Si $x$ e $y$ son coprimos, entonces $a$ es el inverso
multiplicativo de $x$ en $\ZZ/y\ZZ$ y $b$ es el inverso multiplicativo de
$y$ en $\ZZ/x\ZZ$, es decir, $[a]_y[x]_y=[1]_y$ y $[b]_x[y]_x=[1]_x$.

\bigskip

Los elementos $[x]_N\in\ZZ/N\ZZ$ que tienen inversa multiplicativa corresponden
unívocamente con los enteros $1\leq x\leq N-1$ coprimos con $N$. En el
párrafo de arriba probamos que esa condición es suficiente. La necesidad
es inmediata, ya que si $\gcd(x,N)=g>1$, entonces $ax\bmod{N}$ será
múltiplo de $g$ para cualquier $a$ y por lo tanto distinto de $1$. Los
elementos invertibles de $\ZZ/N\ZZ$ forman un grupo (con respecto a la multiplicación) que se denota $(\ZZ/N\ZZ)^\ast$ o también $\UU(\ZZ/N\ZZ)$
llamado el grupo de unidades de $\ZZ/N\ZZ$. La cantidad de elementos de este
grupo se denota $\varphi(N)$.
\[
\varphi(N) = \left|(\ZZ/N\ZZ)^*\right|=
\left|\left\{x\,:\,1\leq x<N,\;\gcd(x,N)=1\right\}\right|
\]
Por ejemplo, si $N$ es primo tenemos que $\varphi(N)=N-1$, ya que todos
los enteros entre $1$ y $N-1$ son coprimos con $N$. Esto nos dice que $\ZZ/N\ZZ$
es un cuerpo, ya que es un anillo (conmutativo) en el que todo elemento no nulo
es invertible.

\bigskip

Todavía no tenemos la suficiente teoría para ver como es la estructura de
$(\ZZ/N\ZZ)^*$ en general, pero sí podemos ver algunos resultados básicos.
Supongamos que $N=p_1^{n_1}p_2^{n_2}\cdots p_k^{n_k}$ con $p_1,\ldots,p_k$
primos distintos y $n_1,\ldots,n_k\geq 1$. Por el teorema~\ref{teor_chino}
tenemos que
\[
   (\ZZ/N\ZZ)^*\simeq (\ZZ/p_1^{n_1}\ZZ)^*\oplus\cdots\oplus(\ZZ/p_k^{n_k}\ZZ)^*
\]
y en particular
\[
    \varphi(N)=\varphi(p_1^{n_1})\cdots\varphi(p_k^{n_k})
\]
por lo que bastará estudiar el caso $N=p^n$ para cierto primo $p\geq2$ y
$n\geq 1$.

\bigskip

Claramente se tiene $\varphi(p^n)=p^n-p^{n-1}$, ya que hay exactamente $p^{n-1}$
enteros $1\leq x<p^n$ divisibles por $p$. Entonces:
\[
   \varphi(N)=\varphi\left(p_1^{n_1}\cdots p_k^{n_k}\right)=N
   \left(1-\frac1{p_1}\right)\cdots\left(1-\frac1{p_k}\right).
\]

Para terminar vamos a ver un resultado clásico de teoría de números.

\begin{teor}[Euler-Fermat]\label{teor-euler-fermat}
Sea $N\geq 1$ y sea $a$ coprimo con $N$. Entonces $a^{\varphi(N)}\equiv 1\pmod{N}$.
En particular, si $N$ es primo y $N\nmid a$, entonces $a^{N-1}\equiv 1\pmod{N}$.
\end{teor}
\begin{proof}
Sea $S=\langle\,[a]\,\rangle$ el subgrupo de $(\ZZ/N\ZZ)^*$ generado por $[a]$.
Claramente se tiene que $S=\left\{[1],[a],[a^2],\ldots,[a^{s-1}]\right\}$ donde
$s=|S|$ es el menor exponente tal que $[a^s]=[1]$. Por el teorema de Lagrange,
sabemos que $|S|\mid|(\ZZ/N\ZZ)^*|$, es decir, $s\mid\varphi(N)$, y por lo tanto
$[a^{\varphi(N)}]=[1]$ como queriamos probar.
\end{proof}
\begin{proof}[Demostración clásica]
Sea $C=\{1\leq x\leq N-1\,:\;\gcd(x,N)=1\}$. Como $a$ es coprimo con $N$, y en
particular es invertible módulo $N$, resulta que la aplicación $m_a:C\to C$
dada por $m_a(x)=ax\bmod N$ es una biyección. Entonces
\[
\prod_{x\in S}ax\equiv\prod_{x\in S}x\pmod{N}
\]
y por lo tanto
\[
a^{\varphi(N)}\prod_{x\in S}x\equiv\prod_{x\in S}x\pmod{N}.
\]
Como todos los elementos de $S$ son invertibles módulo $N$, podemos quitarlos
de ambos miembros de la identidad de arriba, y entonces $a^{\varphi(N)}\equiv 1
\pmod{N}$.
\end{proof}

\begin{prob}
Mostrar que $\varphi(100)=40$. De acuerdo con el teorema de Euler-Fermat
se tiene que $a^{40}\equiv 1\pmod{100}$ para cualquier $a$ coprimo con $100$.
Escribir un programa en Python3 que verifique esa afirmación.
Calcular la estructura del grupo $(\ZZ/100\ZZ)^*$ determinando manualmente
la estructura de $(\ZZ/4\ZZ)^*$ y $(\ZZ/25\ZZ)^*$. Demostrar que $a^{20}\equiv 1
\pmod{100}$ para todo $a$ coprimo con $100$ y modificar el programa anterior
para verficar esta nueva afirmación.
\end{prob}

\begin{prob}
Calcular (a mano) el resto de dividir $10^{10000000}$ por
$2\cdot3\cdot5\cdot7\cdot11\cdot 13$ y luego verificarlo haciendo
\texttt{(10 ** 10000000) \% (2*3*5*7*11*13)} en Python3. Comparar
la velocidad de esta instrucción con
\texttt{potencia\_mod(10, 10000000, 2*3*5*7*11*13)}.
\end{prob}

\begin{prob}\label{prob:carmichael}
Un entero $N\geq 2$ se dice pseudoprimo de Fermat en base $a$, para $a\in\ZZ$
coprimo con $N$, si $a^{N-1}\equiv 1\pmod{N}$. Se dice que $N$ es pseudoprimo
de Fermat fuerte si es pseudoprimo en cualquier base $a$ coprimo con $N$.
El teorema de Euler-Fermat muestra que todo primo es pseudoprimo de Fermat
fuerte, pero lamentablemente la recíproca no es cierta. Los números compuestos
$N\geq 2$ que son pseudoprimos de Fermat fuertes se llaman números de Carmichael.
Escribir un programa en Python3 que determine los $10$ primeros números de
Carmichael.
\end{prob}

\bigskip

En lo que sigue vamos a estudiar en detalle la estructura de los grupos
de unidades $(\ZZ/N\ZZ)^*$.

\bigskip

Recordemos primero que todo grupo abeliano finitamente generado $G$ es
isomorfo a
\[
   \ZZ^r\oplus\ZZ/d_1\ZZ\oplus\ZZ/d_2\ZZ\oplus\cdots\oplus\ZZ/d_k\ZZ
\]
para un cierto $r\geq 0$ (el rango del grupo) y ciertos $d_1,\ldots,d_k\geq 1$
tales que $d_i\mid d_{i+1}$ para todo $i$. Lógicamente, en la escritura
de arriba, cada uno de los sumandos $\ZZ/d_i\ZZ$ es visto como grupo
abeliano \emph{aditivo} y no como anillo. Para el caso de grupos abelianos
finitos, se tiene que $r=0$. Si además el grupo es cíclico, se tiene $k=1$,
es decir, $G\simeq\ZZ/d_1\ZZ$ para cierto $d_1\geq 1$.

\bigskip

El orden de un elemento $a$ en un grupo $G$ es el menor $n\geq 1$ tal que
\[n\cdot a=\underbracket{a+a+\cdots+a}_{n\;\text{veces}}=0_G,\] o
equivalentemente, la cantidad de elementos del subgrupo generado por $a$.
Por el teorema de Lagrange, se tiene siempre que $ord(a)\mid|G|$. En el caso
del grupo multiplicativo $(\ZZ/N\ZZ)^*$, el orden de un elemento $[a]_N$ es el
menor $n\geq 1$ tal que $a^n\equiv1\pmod{N}$, y como mencionamos arriba, se
tiene que $ord([a]_N)\mid\varphi(N)$. Por abuso de notación, en los casos donde
el valor de $N$ sea claro por el contexto, escribiremos $ord(a)$ en lugar
de $ord([a]_N)$. Por supuesto, esto solo tiene sentido cuando $\gcd(a,N)=1$.

\bigskip

Ahora estamos en condiciones de entender la estructura de los grupos
$(\ZZ/N\ZZ)^*$ completamente. Por supuesto, basta con reducirse al
caso $N=p^n$ para $p\geq 2$ primo y $n\geq 1$.

\begin{teor}\label{teor-estr}
Sea $p\geq 2$ primo y $n\geq 1$. Entonces $(\ZZ/p^n\ZZ)^*$ es cíclico para
cualquier $p$ impar y para el caso $p=2$, $n\leq 2$, es decir, es isomorfo
a $\ZZ/\varphi(p^n)\ZZ$. Para $n\geq 3$, el grupo $(\ZZ/2^n\ZZ)^*$ es
isomorfo a $\ZZ/2\ZZ\oplus\ZZ/2^{n-2}\ZZ$.
\end{teor}

La demostración del teorema~\ref{teor-estr} es larga y la vamos a dividir
en varias partes.

\begin{prop}\label{parte1}
Sea $p\geq 2$ primo. Entonces $(\ZZ/p\ZZ)^*$ es cíclico.
\end{prop}
\begin{proof}
Supongamos que $(\ZZ/p\ZZ)^*$ es isomorfo a $\ZZ/d_1\ZZ\oplus\cdots\ZZ/d_k\ZZ$
para ciertos $d_1,\ldots,d_k\geq 2$ tales que $d_i\mid d_{i+1}$ para todo
$i=1,\ldots,k-1$. Esto implica que $x^{d_k}=1$ para todo $x\in(\ZZ/p\ZZ)^*$,
o equivalentemente, que el polinomio $X^{d_k}-1\in(\ZZ/p\ZZ)[X]$ tiene como
raíces a todos los elementos de $(\ZZ/p\ZZ)^*$. Como sabemos que $\ZZ/p\ZZ$ es
un cuerpo, el polinomio puede tener como mucho $d_k$ raíces, y entonces
$d_k\geq \varphi(p)=p-1$. Por otra parte, sabemos que
$d_1d_2\cdots d_k=\varphi(p)=p-1$, y entonces $d_k\leq p-1$. La única forma de
que esto es posible es si $k=1$ y $d_1=p-1$, es decir, el grupo $(\ZZ/p\ZZ)^*$
es isomorfo al grupo cíclico $\ZZ/(p-1)\ZZ$.
\end{proof}

\begin{lema}\label{parte2}
Sean $p\geq 3$ primo, $n\geq 1$ y $b=1+p^nu$ para cierto $u\in\ZZ$ tal que
$p\nmid u$. Entonces $b^p=1+p^{n+1}v$ con $p\nmid v$.
\end{lema}
\begin{proof}
Usando la fórmula del binomio, nos queda
\[
   b^p=(1+p^nu)^p=
   \sum_{i=0}^p\binom{p}{i}p^{ni}u^i=
   1+p^{n+1}u+\sum_{i=2}^p\binom{p}{i}p^{ni}u^i=
   1+p^{n+1}u+p^{n+2}t
\]
ya que $p^{n+2}\mid\binom{p}{i}p^{ni}$ para cualquier $i\geq 2$. La conclusión
se sigue de tomar $v=u+pt$.
\end{proof}

\begin{lema}\label{parte3}
Sean $p\geq 3$ primo y $a\in\ZZ$ tal que $p\nmid a$ y
$a^{p-1}\not\equiv1\pmod{p^2}$. Si $ord([a]_p)=p-1$, entonces
$ord([a]_{p^n})=(p-1)p^{n-1}$ para todo $n\geq 2$.
\end{lema}
\begin{proof}
Por el teorema de Euler-Fermat sabemos que $a^{p-1}\equiv1\pmod{p}$, es decir,
$a^{p-1}=1+pu$ para cierto $u\in\ZZ$. Por otra parte, debe ser $p\nmid u$,
ya que de lo contrario tendríamos que $a^{p-1}\equiv1\pmod{p^2}$, en
contradicción con las hipótesis.

Si $a^{\ell}\equiv1\pmod{p^n}$, también será que $a^{\ell}\equiv1\pmod{p}$
y por lo tanto $a^{\gcd(\ell,p-1)}\equiv1\pmod{p}$. Teniendo en cuenta
que $ord([a]_{p})=p-1$, la única posibilidad es que $p-1\mid\ell$. En
particular, esto prueba que $ord([a]_{p^n})$ es un múltiplo de
$p-1$. Como también sabemos que $ord([a]_{p^n})$ divide a
$\varphi(p^n)=(p-1)p^{n-1}$, nos queda que $ord([a]_{p^n})=(p-1)p^k$ para
cierto $1\leq k\leq n-1$.

Aplicando $k$ veces consecutivas el lema~\ref{parte2} al entero $b=a^{p-1}=1+pu$,
nos queda que $a^{(p-1)p^k}=1+p^{k+1}v$ para cierto $v\in\ZZ$ tal que $p\nmid v$.
Por otra parte, debe ser $a^{(p-1)p^k}\equiv1\pmod{p^n}$ y por lo tanto $k=n-1$
como queríamos probar.
\end{proof}

\begin{prop}\label{parte4}
Sean $p\geq 3$ primo y $n\geq 1$. Entonces $(\ZZ/p^n\ZZ)^*$ es cíclico.
\end{prop}
\begin{proof}
En la proposición~\ref{parte1} ya vimos el caso $n=1$, es decir, ya sabemos
que $(\ZZ/p\ZZ)^*$ es cíclico. Tomemos $a\in\ZZ$ tal que $p\nmid a$ y que
$[a]_p$ sea un generador de $(\ZZ/p\ZZ)^*$, es decir, que $ord([a]_p)=p-1$.
Por supuesto, tenemos que $a^{p-1}=1+pu$ para cierto $u\in\ZZ$. En el caso
de que $p\nmid u$, el lema~\ref{parte3}, prueba que $[a]_{p^n}$ es un
generador de $(\ZZ/p^n\ZZ)^*$ para todo $n\geq 2$.

En el caso de que $p\mid u$, es decir, que $a^{p-1}\equiv 1\pmod{p^2}$, podemos
cambiar a $a$ por $a+p$ y volver a empezar. En efecto, se puede ver que
\[
   (a+p)^{p-1}=1+pu+(p-1)pa^{p-2}+\sum_{i=2}^{p-1}\binom{p-1}{i}a^{p-1-i}p^i
\]
es de la forma $1+pu'$ con $p\nmid u'$, ya que $pu$ y todos lo términos con
$i\geq 2$ son múltiplos de $p^2$, pero el término $(p-1)a^{p-2}p$ es divisible
por $p$ pero no $p^2$.
\end{proof}

Un entero $a\in\ZZ$ tal que $p\nmid a$ y $ord([a]_{p^n})=\varphi(p^n)$ se
llama una raíz primitiva módulo $p^n$. La proposición anterior muestra la
existencia de raíces primitivas para $p$ primo impar y $n\geq 1$.

\begin{lema}\label{parte5}
Sea $n\geq 2$. Entonces $ord([5]_{2^n})=2^{n-2}$.
\end{lema}
\begin{proof}
Sabemos que $ord([5]_{2^n})\mid\varphi(2^n)=2^{n-1}$ y por lo tanto será de
la forma $2^k$ para cierto $k\geq 0$. Vamos a probar inductivamente que
$5^{2^k}=1+2^{k+2}u_k$ con $u_k$ impar. Para $k=0$ es inmediato tomando
$u_0=1$. Ahora veamos el caso $k+1$ suponiendo probado el caso $k$.
\[
  5^{2^{k+1}}=(1+2^{k+2}u_k)^2=1+2^{k+3}u_k+2^{2k+4}u_k^2=
  1+2^{k+3}(u_k+2^{k+1}u_k^2)
\]
Basta con definir $u_{k+1}=u_k+2^{k+1}u_k^2$. Esto completa la inducción.
De este resultado, se puede ver que el menor $k\geq 0$ que hace que $5^{2^k}\equiv1
\pmod{2^n}$ es $k=n-2$, como queríamos probar.
\end{proof}

\begin{lema}\label{parte6}
Sea $n\geq 3$ y sea $a\in\ZZ$ impar. Entonces $ord([a]_{2^n})\mid 2^{n-2}$.
\end{lema}
\begin{proof}
El enunciado es equivalente a decir que $a^{2^{n-2}}\equiv1\pmod{2^n}$ para
todo $n\geq 3$ y $a$ impar. Procederemos por inducción. El caso $n=3$ es
inmediato, ya que $(1+2r)^2=1+4r+4r^2=1+4r(r+1)\equiv 1\pmod{8}$ para
cualquier $r\in\ZZ$. Para al paso inductivo, supongamos que ya tenemos el caso
$n$ probado, es decir, que $a^{2^{n-2}}=1+2^nu$ para cierto $u\in\ZZ$. Elevando
al cuadrado nos queda
$a^{2^{n-1}}=(1+2^nu)^2=1+2^{n+1}u+2^{2n}u^2\equiv1\pmod{2^{n+1}}$, lo
que demuestra el caso $n+1$.
\end{proof}

\begin{proof}[Demostración del teorema~\ref{teor-estr}]
El caso $p$ impar fue probado en la proposición~\ref{parte4}, por lo que solo
falta ver el caso $p=2$. Los casos $n=1$ y $n=2$ se puede comprobar manualmente.
Para $n\geq 3$, digamos que $(\ZZ/2^n\ZZ)^*\simeq\ZZ/d_1\ZZ\oplus\cdots\oplus
\ZZ/d_k\ZZ$ con $d_i\mid d_{i+1}$ para $i=1,2,\ldots,k-1$. El lema~\ref{parte6}
demuestra que $d_k\mid 2^{n-2}$ y el lema~\ref{parte5} que $2^{n-2}\mid d_k$.
En particular, debe ser $d_k=2^{n-2}$. Por otra parte, tenemos que $d_1\cdots
d_k=\varphi(2^n)=2^{n-1}$. La única forma de que esto se cumpla es si $k=2$
y $d_1=2$ y por lo tanto $(\ZZ/2^n\ZZ)^*\simeq\ZZ/2\ZZ\oplus\ZZ/2^{n-2}\ZZ$.
\end{proof}

En el caso $p$ primo, sabemos que $(\ZZ/p\ZZ)^*\simeq\ZZ/(p-1)\ZZ$. El
isomorfismo (de grupos) se puede escribir de forma explícita: tomamos
$a\in\ZZ$ una raíz primitiva módulo $p$ y definimos
\[
\begin{aligned}
\exp:\ZZ/(p-1)\ZZ  &\rightarrow(\ZZ/p\ZZ)^* \\
               [k]_{p-1} &\mapsto [a^k]_p
\end{aligned}
\]
El morfismo inverso $\log:(\ZZ/p\ZZ)^*\rightarrow\ZZ/(p-1)\ZZ$ se llama
\emph{logaritmo discreto} (en base $a$ y módulo $p$ si es necesario aclarar).
El programa~\ref{prog51} muestra que $\exp([k])$ puede calcularse eficientemente.
Sin embargo, no se conoce actualmente ningún algoritmo de complejidad binaria
polinomial para calcular logaritmos discretos. Esta particularidad se usa
extensivamente en protocolos criptográficos.

\bigskip

A continuación vamos a ver tres aplicaciones interesantes de las raíces
primitivas: el test de primalidad de Lucas, el criterio de Korselt para
números de Carmichael y el método de factorización $p-1$ de Pollard.

\begin{teor}[Test de primalidad de Lucas]
Para cualquier $N\geq 2$ son equivalentes:
\begin{enumerate}
\item $N$ es primo.
\item Para todo primo $p$ que divide a $N-1$, existe $a\in\ZZ$ tal que
$a^{N-1}\equiv1\pmod{N}$ y $a^{(N-1)/p}\not\equiv1\pmod{N}$.
\end{enumerate}
\end{teor}
\begin{proof}
$(1\Rightarrow 2)$: Elijamos $[a]$ un generador de $(\ZZ/N\ZZ)^*$,
es decir, $a$ es una raíz primitiva módulo $N$. Por el teorema de Euler-Fermat
sabemos que $a^{N-1}\equiv1\pmod{N}$ y por ser raíz primitiva $N-1$ es el
menor exponente $i$ tal que $a^i\equiv1\pmod{N}$. En particular,
$a^{(N-1)/p}\not\equiv1\pmod{N}$ para cualquier $p\mid N-1$.

$(2\Rightarrow 1)$: Sea $p$ un primo que divide a $N-1$ y sea $a$ tal
que $a^{N-1}\equiv1\pmod{N}$ y $a^{(N-1)/p}\not\equiv1\pmod{N}$. Esto nos
dice que el orden multiplicativo de $[a]$ en $(\ZZ/N\ZZ)^*$ es un divisor
de $N-1$ pero no de $(N-1)/p$. Esto implica que la máxima potencia de $p$
que divide a $N-1$ tiene que dividir al orden de $[a]$ y en particular a
$\varphi(N)=|(\ZZ/N\ZZ)^*|$. Repitiendo esto para todos los primos $p$ que
dividen a $N-1$, concluímos que $N-1\mid\varphi(N)$ y por lo tanto
$\varphi(N)=N-1$. Esto solo es posible si $N$ es primo.
\end{proof}

\begin{teor}[Criterio de Korselt]\label{teor-korselt}
Un entero $N\geq 2$ es pseudoprimo fuerte de Fermat (ver
problema~\ref{prob:carmichael}) si y solo si $N$ es libre de cuadrados (todos
los factores primos de $N$ tienen multiplicidad $1$) y $p-1\mid N-1$ para todo
factor primo $p$ de $N$.
\end{teor}
\begin{proof}
Supongamos que $N=p_1p_2\cdots p_k$ con los $p_i$ primos distintos y tal que
$p_i-1\mid N-1$ para $i=1,\ldots,k$. Sea $a\in\ZZ$ coprimo con $N$, es decir,
$p_i\nmid a$ para todo $i$. Por el teorema de Euler-Fermat tenemos que
$a^{p_i-1}\equiv 1\pmod{p_i}$ y como $p_i-1\mid N-1$ también
$a^{N-1}\equiv1\pmod{p_i}$ para todo $i$. Por lo tanto $a^{N-1}\equiv1\pmod{N}$,
es decir, $N$ es pseudoprimo de Fermat en base $a$. Como esto vale para cualquier
$a$, concluímos que $N$ es un pseudoprimo de Fermat fuerte.

Ahora veamos la recíproca. Supongamos que $N=p_1^{n_1}p_2^{n_2}\cdots p_k^{n_k}$
es pseudoprimo de Fermat fuerte, donde los $p_i$ son primos distintos y los
$n_i\geq 1$. Como $N$ debe ser pseudoprimo en base $a=-1$, tenemos que
$(-1)^{N-1}\equiv1\pmod{N}$ y por lo tanto $N$ debe ser impar o bién $N=2$, en
cuyo caso ya hemos terminado. A partir de ahora supondremos que $N$ es impar.
En primer lugar, tenemos que probar que los $n_i$ no pueden ser mayores que $1$.
Supongamos que $n_i\geq 2$ para algún $i$ y tomemos $a$ una raíz primitiva módulo
$p_i^2$. Por hipótesis sabemos que $a^{N-1}\equiv1\pmod{p_i^2}$ y por ser
$a$ primitiva, debe ser $\varphi(p_i^2)\mid N-1$. Pero esto es imposible, ya
que $\varphi(p_i^2)=p_i(p_i-1)$ es divisible por $p_i$ pero $N-1$ no lo es. Esta
contradicción muestra que $n_1=n_2=\cdots=n_k=1$, es decir, que $N$ es libre de 
cuadrados. Por último, tomemos $a$ una raíz primitiva módulo $p_i$. Como
$a^{N-1}\equiv 1\pmod{p_i}$ por hipótesis y $a$ es primitiva, tenemos también
que $p_i-1\mid N-1$ como queríamos probar.
\end{proof}

\begin{defi}\label{def:suave}
Sea $B\geq 2$. Un entero $n\neq0$ se dice $B$-suave ($B$-smooth) si solo
es divisible por primos menores o iguales que $B$.
\end{defi}

\bigskip

Supongamos que tenemos un entero $N\geq 2$ compuesto al que queremos factorizar
como producto de dos factores no triviales. El método $p-1$ de Pollard permite
hacer esto suponiendo que $N$ tiene al menos dos factores primos $p$ y $q$
distintos, tales que $p-1$ es $B$-suave (para un $B\geq2$ conocido) y que $q-1$
no es $B$-suave. Si llamamos
\[
   \beta=\prod_{\myatop{p\;\text{primo}}{p\leq B}}p^{\lceil\log_pN\rceil}
\]
la condición de que $p-1$ es $B$-suave es equivalente a decir que $p-1\mid\beta$.
El método de Pollard es el siguiente: se elige un $a\in[1,N-1]$ uniformemente al
azar, se comprueba que $x=\gcd(a,N)=1$ y finalmente se calcula
$y=\gcd(a^{\beta}-1,N)$. En caso de que $x\neq 1$, ya hemos resuelto el
problema, ya que $x\mid N$ y $1<x\leq a\leq N-1$. Queda por ver que sucede en
el caso $x=1$, es decir, cuando $a$ es coprimo con $N$. Como $a$ será también
coprimo con $p$, tenemos que $a^{p-1}\equiv1\pmod{p}$ por el teorema de
Euler-Fermat, y por lo tanto $a^{\beta}\equiv1\pmod{p}$. Esto nos
dice que $y=\gcd(a^{\beta}-1,N)$ será divisible por~$p$. Veamos que es
bastante improbable que $y$ sea divisible por~$q$. En efecto, la cantidad de
soluciones de la congruencia $a^\beta\equiv1\pmod{q}$ con $\gcd(a,q)=1$ es
$\gcd(\beta,q-1)<\frac{q-1}B$, ya que hay al menos un factor primo de
$q-1$ mayor que $B$ y que por lo tanto no aparece en $\beta$. Eso nos dice que
la probabilidad condicional de que $a^\beta\equiv1\pmod{q}$ sabiendo que
$\gcd(a,N)=1$ es menor que $1/B$. En caso de que $y$ no produzca un factor
no trivial, es decir, que $y=N$, bastará con repetir el experimento.

\begin{prob}\label{prob-menor-rp}
Implementar la función \texttt{menor\_raiz\_primitiva(p)} que obtenga la
menor raíz primitiva $1\leq a\leq p-1$ para un primo $p$ dado.
\end{prob}

\begin{prob}
Calcular la estructura de $(\ZZ/N\ZZ)^*$ para $N=100, 1000, 10!$. ¿Cuál es
el menor $n$ tal que $a^n\equiv 1\pmod{N}$ para todo $a$ coprimo con $N$?
\end{prob}

\begin{prob}
Encontrar todos los elementos de $(\ZZ/2^n\ZZ)^*$ de orden $2$. Demostrar
que la aplicación
\[
\begin{aligned}
\ZZ/2\ZZ\oplus\ZZ/2^{n-2}\ZZ & \rightarrow(\ZZ/2^n\ZZ)^* \\
([a],[b]) & \mapsto [(-1)^a5^b]
\end{aligned}
\]
es un isomorfismo de grupos para todo $n\geq 3$.
\end{prob}

\begin{prob}
Demostrar que en un grupo cíclico finito $G$ hay $\varphi(|G|)$ elementos
de orden $|G|$. ¿Cuántas raíces primitivas hay módulo un primo $p$?
\end{prob}

\begin{prob}
Para cada uno de los primeros $10$ primos $p=2,3,5,7,\ldots$ determinar
el orden multiplicativo de $2$ módulo $p$ y obtener condiciones necesarias
y suficientes sobre $n$ para que $p\mid 3\cdot2^n+1$.
\end{prob}

\begin{prob}\label{prob-p}
Mostrar, con la ayuda de Python3, que $N=3\cdot 2^{189}+1$ es primo
utilizando el test de primalidad de Lucas. Para cada $p\in\{2,3\}$, determinar
la cantidad de enteros $1\leq a<N$ tales que $a^{(N-1)/p}\not\equiv1\pmod{N}$.
¿Es posible encontrar $a$ tal que $a^{(N-1)/2}\not\equiv1\pmod{N}$ y
$a^{(N-1)/3}\not\equiv1\pmod{N}$?
\end{prob}

\begin{prob}
Sea $p$ primo y sea $e\geq1$ tal que $\gcd(e,p-1)=1$. Demostrar que la
aplicación
\[
\begin{aligned}
   (\ZZ/p\ZZ)^* & \rightarrow(\ZZ/p\ZZ)^* \\
   [x] &\mapsto [x^e]
\end{aligned}
\]
es un isomorfismo de grupos.
\end{prob}

\begin{prob}
¿Cuántas soluciones $(x,y)\in\ZZ/p\ZZ$ tiene la ecuación
$y^2\equiv x^3+7\pmod{p}$ para un primo $p\equiv 2\pmod{3}$?
Modificar el programa~\texttt{curva\_eliptica\_mod13.py} del tema~\#1 para comprobarlo
para los primeros $100$ primos.
\end{prob}

\begin{prob}
Utilizar el criterio de Korselt para probar que si $6k+1$, $12k+1$ y $18k+1$
son primos (para un cierto entero $k\geq 1$), entonces $N=(6k+1)(12k+1)(18k+1)$
es un número de Carmichael. Escribir un programa en Python3 para encontrar
los primeros $5$ números de Carmichael de esta forma.
\end{prob}

\begin{prob}
Utilizar el método $p-1$ de Pollard para factorizar
\[
  N=1533639298078728959641946066766007559698078324136928757330809029848359733271767
\]
utilizando $B=50$. ¿Cómo se puede calcular $\gcd(a^\beta-1,N)$ de forma
eficiente?
\end{prob}

\end{document}