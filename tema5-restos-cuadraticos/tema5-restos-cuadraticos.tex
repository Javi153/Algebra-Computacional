\documentclass[a4paper, 11pt]{article}

\usepackage[utf8]{inputenc}
\usepackage{amsmath,amssymb,amsthm}
\usepackage[margin=2cm]{geometry}
\usepackage{listings}
\usepackage{color}
\usepackage{mathtools}

\definecolor{deepblue}{rgb}{0,0,0.5}
\definecolor{deepred}{rgb}{0.6,0,0}
\definecolor{deepgreen}{rgb}{0,0.5,0}
\definecolor{lightgrey}{rgb}{0.95,0.95,0.95}

\renewcommand*{\lstlistingname}{Programa}

\lstset{
   language=Python,
   basicstyle=\ttfamily\small,
   keywordstyle=\color{deepblue}\bfseries\itshape,
   commentstyle=\color{deepgreen}\itshape,
   stringstyle=\color{deepred},
   backgroundcolor=\color{lightgrey},
   morekeywords={as,assert,nonlocal,with,yield},
   showstringspaces=false,
   numbers=left,
   rulecolor=\color{black},
   captionpos=b,
   frame=leftline,
   literate=
   {á}{{\'a}}1 {é}{{\'e}}1 {í}{{\'i}}1 {ó}{{\'o}}1 {ú}{{\'u}}1
   {Á}{{\'A}}1 {É}{{\'E}}1 {Í}{{\'I}}1 {Ó}{{\'O}}1 {Ú}{{\'U}}1
   {à}{{\`a}}1 {è}{{\`e}}1 {ì}{{\`i}}1 {ò}{{\`o}}1 {ù}{{\`u}}1
   {À}{{\`A}}1 {È}{{\`E}}1 {Ì}{{\`I}}1 {Ò}{{\`O}}1 {Ù}{{\`U}}1
   {ä}{{\"a}}1 {ë}{{\"e}}1 {ï}{{\"i}}1 {ö}{{\"o}}1 {ü}{{\"u}}1
   {Ä}{{\"A}}1 {Ë}{{\"E}}1 {Ï}{{\"I}}1 {Ö}{{\"O}}1 {Ü}{{\"U}}1
   {â}{{\^a}}1 {ê}{{\^e}}1 {î}{{\^i}}1 {ô}{{\^o}}1 {û}{{\^u}}1
   {Â}{{\^A}}1 {Ê}{{\^E}}1 {Î}{{\^I}}1 {Ô}{{\^O}}1 {Û}{{\^U}}1
   {ã}{{\~a}}1 {ẽ}{{\~e}}1 {ĩ}{{\~i}}1 {õ}{{\~o}}1 {ũ}{{\~u}}1
   {Ã}{{\~A}}1 {Ẽ}{{\~E}}1 {Ĩ}{{\~I}}1 {Õ}{{\~O}}1 {Ũ}{{\~U}}1
   {œ}{{\oe}}1 {Œ}{{\OE}}1 {æ}{{\ae}}1 {Æ}{{\AE}}1 {ß}{{\ss}}1
   {ű}{{\H{u}}}1 {Ű}{{\H{U}}}1 {ő}{{\H{o}}}1 {Ő}{{\H{O}}}1
   {ç}{{\c c}}1 {Ç}{{\c C}}1 {ø}{{\o}}1 {Ø}{{\O}}1 {å}{{\r a}}1 {Å}{{\r A}}1
   {€}{{\euro}}1 {£}{{\pounds}}1 {«}{{\guillemotleft}}1
   {»}{{\guillemotright}}1 {ñ}{{\~n}}1 {Ñ}{{\~N}}1 {¿}{{?`}}1 {¡}{{!`}}1 
}

\newcommand{\NN}{\mathbb{N}}
\newcommand{\ZZ}{\mathbb{Z}}
\newcommand{\QQ}{\mathbb{Q}}
\newcommand{\RR}{\mathbb{R}}
\newcommand{\CC}{\mathbb{C}}
\newcommand{\FF}{\mathbb{F}}
\newcommand{\UU}{\mathcal{U}}

\newcounter{numerodetema}
\newcommand\tema[1]{{\eject\stepcounter{numerodetema}\bf Álgebra Computacional
2023/24 \hfill Tema~\#\arabic{numerodetema}}\par\smallskip\hrule\bigskip\par
\begin{center}{\Large\bf #1}\end{center}}

\theoremstyle{plain}
\newtheorem{teor}{Teorema}[numerodetema]
\newtheorem{coro}[teor]{Corolario}
\newtheorem{lema}[teor]{Lema}
\newtheorem{prop}[teor]{Proposición}
\theoremstyle{definition}
\newtheorem{defi}[teor]{Definición}
\newtheorem{prob}{Problema}[numerodetema]

\newcommand\myatop[2]{\genfrac{}{}{0pt}{}{#1}{#2}}

\parindent=0cm

\begin{document}

\setcounter{numerodetema}{4}
\tema{Restos cuadráticos y tests aleatorios de no-primalidad}

Vamos a estudiar las ecuaciones de la forma $x^2\equiv a\pmod{N}$
para $a$ y $N$ dados. Por el teorema chino del resto, bastará con considerar
el caso $N=p^n$ para $p\geq 2$ primo y $n\geq 1$.

\bigskip

El primer caso que vamos a discutir es cuando $N$ es un primo impar. Sabemos que en
este caso $(\ZZ/N\ZZ)^*\simeq\ZZ/(N-1)\ZZ$ y por lo tanto hay exactamente $(N-1)/2$
elementos de $(\ZZ/N\ZZ)^*$ que son cuadrados. Dicho de forma más complicada,
el morfismo de grupos
\[
\begin{aligned}
   sqr:(\ZZ/N\ZZ)^* &\rightarrow(\ZZ/N\ZZ)^* \\
   [x] &\mapsto [x^2]
\end{aligned}
\]
tiene ${\rm Ker}(sqr)=\{[1],[-1]\}$ de dos elementos y por lo tanto
su imagen ${\rm Im}(sqr)\simeq (\ZZ/N\ZZ)^*/{\rm Ker}(sqr)$ es de $(N-1)/2$
elementos. El subgrupo ${\rm Im}(sqr)$ está formado por los \emph{restos
cuadráticos} módulo $N$. Si $[a]\in{\rm Im}(sqr)$, entonces la ecuación
$x^2\equiv a\pmod{N}$ tiene exactamente dos soluciones $1\leq x<N$ y si
$[a]\not\in{\rm Im}(sqr)$ la ecuación $x^2\equiv a\pmod{N}$ no tiene
ninguna solución.

\bigskip

No hemos incluído en la discusión a la ecuación $x^2\equiv0\pmod{N}$, ya que
para $N$ primo la única solución es $x\equiv 0\pmod{N}$ y es inmediata de
encontrar.

\bigskip

A nivel computacional tenemos que analizar las siguientes dos preguntas:
\begin{description}
\item[(REC)] ¿Cómo reconocer eficientemente si un $a\in\ZZ$ coprimo con $N$ es
un resto cuadrático módulo~$N$?
\item[(SOL)] Suponiendo que $a\in\ZZ$ coprimo con $N$ es un resto cuadrático
módulo $N$, ¿cómo obtener las dos soluciones de la ecuación $x^2\equiv a\pmod{N}$
eficientemente?
\end{description}
La eficiencia es clave en las preguntas anteriores, ya que si no podríamos
hacer una busqueda exhaustiva de complejidad binaria exponencial en $\log(N)$.

\begin{prop}\label{prop:sqr1}
Sea $N$ un primo impar y sea $a\in\ZZ$ coprimo con $N$. Son equivalentes:
\begin{enumerate}
\item $a$ es un resto cuadrático módulo $N$.
\item $a^{(N-1)/2}\equiv 1\pmod{N}$.
\end{enumerate}
\end{prop}
\begin{proof}
$(1\Rightarrow2)$: Supongamos que $x^2\equiv a\pmod{N}$ para cierto $x\in\ZZ$.
Es claro que $x$ debe ser coprimo con $N$, ya que si no tendríamos
$a\equiv 0\pmod{N}$, en contradicción con las hipótesis.
Entonces $a^{(N-1)/2}\equiv x^{N-1}\equiv 1\pmod{N}$ por el teorema de Euler-Fermat.

$(\neg1\Rightarrow\neg2)$: Hemos probado que los $(N-1)/2$ restos cuadráticos módulo
$N$ son ceros del polinomio $T^{(N-1)/2}-1\in(\ZZ/N\ZZ)[T]$. Como un polinomio (con
coeficientes en un cuerpo) puede tener a lo sumo tantas raíces como el grado,
resulta que para los no restos cuadráticos $a$ se tendrá que
$a^{(N-1)/2}\not\equiv 1\pmod{N}$.
\end{proof}

La proposición~\ref{prop:sqr1} y el algoritmo de la exponenciación binaria modular
resuelven eficientemente la pregunta \textbf{(REC)} de arriba. Basta invocar \texttt{pow(a,(N-1)//2,N)} y ver el resultado para determinar si $a\in\ZZ$
coprimo con $N$ primo impar es un resto cuadrático o no. Una pregunta interesante es:
¿cuál es el resultado de hacer $a^{(N-1)/2}\bmod{N}$ cuando $a$ no es un resto
cuadrático?

\begin{prop}\label{prop:sqr2}
Sea $N$ un primo impar y supongamos que $a\in\ZZ$, coprimo con $N$, no es un resto
cuadrático módulo~$N$. Entonces $a^{(N-1)/2}\equiv-1\pmod{N}$.
\end{prop}
\begin{proof}
Por el teorema de Euler-Fermat sabemos que $\left(a^{(N-1)/2}\right)^2\equiv
1\pmod{N}$. También sabemos que las únicas soluciones de $y^2\equiv1\pmod{N}$ son
$y\equiv\pm1\pmod{N}$, ya que $N\mid y^2-1$ si y solo si $N\mid y-1$ o $N\mid y+1$.
Por la proposición anterior tenemos que $a^{(N-1)/2}\not\equiv1\pmod{N}$ y por lo
tanto debe ser $a^{(N-1)/2}\equiv-1\pmod{N}$.
\end{proof}

\begin{defi}
Sea $N$ un primo impar y sea $a\in\ZZ$. El símbolo de Legendre se define
como
\[
\left(\frac{a}{N}\right)=\left\{\begin{array}{rl} 0 & \text{si}\;N\mid a, \\
1 & \text{si}\;a\;\text{es resto cuadrático módulo}\;N, \\
-1 & \text{si}\;a\;\text{no es resto cuadrático módulo}\;N.
\end{array}\right.
\]
\end{defi}
El símbolo de Legendre solo depende de la clase de $a$ módulo $N$. Las
proposiciones~\ref{prop:sqr1} y~\ref{prop:sqr2} dicen que para un primo impar
$N$ se tiene que
\begin{equation}\label{eq-legendre}
   a^{(N-1)/2}\equiv\left(\frac{a}{N}\right)\pmod{N}.
\end{equation}
Una consecuencia interesante de esta identidad es que $\left(\frac{ab}{N}\right)=
\left(\frac{a}{N}\right)\left(\frac{b}{N}\right)$ para todo $a,b\in\ZZ$.
Un hecho que de no ser por esa identidad sería difícil de ver es que el
producto de dos no restos cuadráticos es un resto cuadrático (módulo un primo $N$).

\bigskip

De la discusión de arriba, se infiere que será importante entender el
valor de $\left(\frac{M}{N}\right)$ para $M,N$ primos. Eso nos daría
alguna información sobre $\left(\frac{a}{N}\right)$ a través de la
factorización de $a$.

\begin{teor}[Ley de la reciprocidad cuadrática y suplementos (Gauss)]
\label{teor-lrc}
Sean $N,M\geq 2$ primos impares. Entonces:
\begin{enumerate}
\item $\left(\frac{M}{N}\right)\left(\frac{N}{M}\right)=(-1)^{(N-1)(M-1)/4}=
\left\{\begin{array}{rl}
1 & \text{si}\;N\equiv1\pmod{4}\;\text{o}\;M\equiv1\pmod{4} \\
-1& \text{si}\;N\equiv3\pmod{4}\;\text{y}\;M\equiv3\pmod{4}
\end{array}\right.$
\item $\left(\frac{2}{N}\right)=(-1)^{(N^2-1)/8}=\left\{\begin{array}{rl}
1 & \text{si}\;N\equiv1,7\pmod{8} \\
-1& \text{si}\;N\equiv3,5\pmod{8}
\end{array}\right.$
\item $\left(\frac{-1}{N}\right)=(-1)^{(N-1)/2}=\left\{\begin{array}{rl}
1 & \text{si}\;N\equiv1\pmod{4} \\
-1& \text{si}\;N\equiv3\pmod{4}
\end{array}\right.$
\end{enumerate}
\end{teor}

Hay más de 240 demostraciones diferentes de la ley de reciprocidad
cuadrática, de las que aquí solo mostraremos una de las más elementales,
que usa ideas parecidas a la demostración clásica del
Teorema de Euler-Fermat. Necesitamos primero el
siguiente lema, que es importante en sí mismo.

\begin{lema}[Einsenstein]\label{lema-eins}
Sea $N\geq2$ un primo impar y sea $a\in\ZZ$ tal que $N\nmid a$. Entonces
$\left(\frac{a}{N}\right)=(-1)^k$ donde $k=\sum_u\lfloor\frac{au}{N}\rfloor$
y la suma recorre todos los enteros $u$ pares entre $2$ y $N-1$, ambos
inclusive. En el caso de que $a$ sea impar, se puede tomar también
$k=\sum_{u=1}^{(N-1)/2}\lfloor\frac{au}N\rfloor$, donde la suma recorre
todos los índices (pares e impares) $u$ entre $1$ y $\frac{N-1}2$.
\end{lema}
\begin{proof}
Sea $U=\{2,4,\ldots,N-1\}$ el conjunto de los números pares entre $2$ y $N-1$.
Consideremos la aplicación $\tau:U\rightarrow U$ dada por
$u\mapsto \pm au \bmod{N}$, donde el signo se elige de modo que el resultado
sea par (esto está bien definido ya que $N$ es impar). Se tiene que $\tau$ es
una biyección, ya que si $\pm au_1\equiv\pm au_2\pmod{N}$ para ciertos
$u_1,u_2\in U$, entonces $u_1\equiv\pm u_2\pmod{N}$ y esto solo es posible
cuando $u_1\equiv u_2\pmod{N}$ ya que $N$ es impar. En efecto, si fuese
$u_1\equiv -u_2\pmod{N}$, tendríamos que $N\mid u_1+u_2$, pero
$4\leq u_1+u_2\leq 2N-2$, por lo que debería ser $u_1+u_2=N$ y eso es imposible
ya que $N$ es impar. Sabiendo que $\tau$ es una biyección, podemos calcular el
producto de todos los números de $U$ de dos formas distintas:
\[
   \prod_{u\in U}u\equiv\prod_{u\in U}\tau(u)\equiv\prod_{u\in U}\pm au
   \equiv (-1)^ra^{(N-1)/2}\prod_{u\in U}u\pmod{N},
\]
donde $r$ es la cantidad de $u\in U$ para los que hemos tenido que usar el
signo $(-)$ al calcular $\tau(u)$. Esta identidad nos dice que
$\left(\frac{a}{N}\right)=(-1)^r$, por lo que solo nos falta probar que
$r$ y $k$ tienen la misma paridad. Para un $u\in U$, sabemos que
$au\bmod{N}=au-N\lfloor\frac{au}{N}\rfloor$, por lo que $au\bmod{N}$ será impar
(y por lo tanto aportará al valor de $r$) si y solo si
$\lfloor\frac{au}{N}\rfloor$ es impar. Esto prueba que
$r\equiv\sum_{u\in U}\lfloor\frac{au}{N}\rfloor=k\pmod{2}$, tal como afirmabamos.
La última parte se sigue de que, cuando $a$ es impar, $\lfloor\frac{au}{N}\rfloor$
y $\lfloor\frac{a(N-u)}{N}\rfloor$ tiene la misma paridad. Eso nos permite cambiar
los índices pares $u\in U$ mayores que $N/2$ por el correspondiente $N-u$ impar
en el rango $1\leq N-u<N/2$.
\end{proof}

\begin{proof}[Demostración del teorema~\ref{teor-lrc}]
1.~Por el lema~\ref{lema-eins} de Einsenstein, tenemos que
$\left(\frac{M}{N}\right)\left(\frac{N}{M}\right)=(-1)^{k+l}$ donde
$k=\sum_{u=1}^{(N-1)/2}\lfloor\frac{uM}{N}\rfloor$ y
$l=\sum_{v=1}^{(M-1)/2}\lfloor\frac{vN}{M}\rfloor$. Es fácil ver que
$k$ es exactamente la cantidad de puntos enteros en el \emph{interior}
del triángulo $(0,0)-(N/2,0)-(N/2,M/2)$ y que, de forma similar,
$l$ es la cantidad de puntos enteros en el interior del triángulo
$(0,0)-(0,M/2)-(N/2,M/2)$. Como $N$ y $M$ son primos, el segmento
$(0,0)-(N/2,M/2)$ no tiene puntos enteros y por lo tanto $k+l$ es la
cantidad de puntos enteros en el interior del rectángulo $(0,0)-(N/2,0)-
(N/2,M/2)-(0,M/2)$, es decir, $(N-1)(M-1)/4$.

2.~Aplicamos el lema~\ref{lema-eins} para $a=2$. Se puede ver que
\[
   k=\sum_{\myatop{u=2}{u\;\text{par}}}^{N-1}\left\lfloor\frac{2u}{N}\right\rfloor=
   |\{u\mid u\;\text{par y}\;\frac{N}{2}<u<N\}|=
   \left\lfloor\frac{N+1}{4}\right\rfloor
\]
y eso es par si $N\equiv1,7\pmod{8}$ e impar si $N\equiv3,5\pmod{8}$.

3.~Ya hemos probado que $\left(\frac{-1}{N}\right)=(-1)^{(N-1)/2}$ y eso
da $1$ si $N\equiv1\pmod{4}$ y $-1$ si $N\equiv3\pmod{4}$.
\end{proof}

Veamos un ejemplo simple de como usar la ley de reciprocidad cuadrática.
Digamos que queremos ver si $2021$ es resto cuadrático módulo el primo~$3331$.
\[
\begin{aligned}
\left(\frac{2021}{3331}\right)&=\left(\frac{43}{3331}\right)\left(\frac{47}{3331}
\right)=\left(\frac{3331}{43}\right)\left(\frac{3331}{47}\right)=\left(\frac{20}{43}
\right)\left(\frac{41}{47}\right)=\left(\frac{2}{43}\right)^2\left(\frac{5}{43}
\right)\left(\frac{41}{47}\right)\\
&=\left(\frac{5}{43}\right)\left(\frac{41}{47}\right)=\left(\frac{43}{5}\right)
\left(\frac{47}{41}\right)=\left(\frac{3}{5}\right)\left(\frac{6}{41}\right)=
\left(\frac{3}{5}\right)\left(\frac2{41}\right)\left(\frac3{41}\right)=
\left(\frac{3}{5}\right)\left(\frac3{41}\right)\\
&=\left(\frac{5}{3}\right)\left(\frac{41}{3}\right)=
\left(\frac{2}{3}\right)\left(\frac{2}{3}\right)=1
\end{aligned}
\]
En efecto, una búsqueda exhaustiva nos muestra que $2021\equiv 534^2\pmod{3331}$,
es decir, $2021$ es un resto cuadrático módulo $3331$ tal como calculamos arriba.

\bigskip

El único problema con este método para calcular el símbolo de Legendre es que
requiere factorizar el numerador en cada paso, y para eso aún no se disponen de
algoritmos eficientes.

\begin{defi}
Sea $N=p_1^{n_1}p_2^{n_2}\cdots p_k^{n_k}$ un entero impar con los $p_i$ primos
y los $n_i\geq 1$. Sea $a\in\ZZ$ coprimo con $N$, es decir, $p_i\nmid a$ para
todo $i=1,\ldots,k$. El símbolo de Jacobi $\left(\frac{a}{N}\right)$ se define
como
\[
  \left(\frac{a}{N}\right)=\prod_{i=1}^k\left(\frac{a}{p_i}\right)^{n_i}
\]
donde los factores de la derecha son símbolos de Legendre. No hay ambigüedad
en la notación ya que ambos símbolos coinciden para $N$ primo.
\end{defi}

El valor de $\left(\frac{a}{N}\right)$ depende solo de la clase de $a$
módulo $N$, al igual que para el símbolo de Legendre. Más aún, Jacobi
demostró que su extensión del símbolo de Legendre hace que la ley de
reciprocidad cuadrática y sus suplementos (teorema~\ref{teor-lrc})
sigan siendo válidos para cualquier $N$ y $M$ impares.

\begin{prob}
Demostrar la ley de reciprocidad cuadrática y sus suplementos para el
símbolo de Jacobi.
\end{prob}

La ventaja del símbolo de Jacobi es que se puede calcular fácilmente
sin necesidad de factorizar el numerador. Se puede utilizar la
técnica con la que implementamos \texttt{gcd\_binario()} o bien la
de \texttt{gcd\_euclides()}. Utilizando este segundo método, el ejemplo
que calculamos antes quedaría
\[
\begin{aligned}
\left(\frac{2021}{3331}\right)&=\left(\frac{3331}{2021}\right)
=\left(\frac{1310}{2021}\right)=
\left(\frac{2}{2021}\right)\left(\frac{655}{2021}\right)=
-\left(\frac{655}{2021}\right)=-\left(\frac{2021}{655}\right)
=-\left(\frac{56}{655}\right) \\
&=-\left(\frac{2}{655}\right)^3\left(\frac{7}{655}\right)
=-\left(\frac{7}{655}\right)=\left(\frac{655}{7}\right)
=\left(\frac{4}{7}\right)=\left(\frac{2}{7}\right)^2=1
\end{aligned}
\]

\begin{prob}
Implementar en Python3 la función recursiva \texttt{jacobi(a,n)} que
calcule el símbolo de Jacobi. ¿Cuál es la complejidad binaria de esta
función?
\end{prob}

{\bf Advertencia 1}: El símbolo de Jacobi NO corresponde con la noción de ser
resto cuadrático módulo un $N$ impar cualquiera. Esa propiedad solo vale
para $N$ primo.

\begin{prob}
Escribir un programa en Python3 que encuentre todos los $a$ que no sean
restos cuadráticos módulo $2021=43\cdot47$, pero que verifiquen
$\left(\frac{a}{2021}\right)=1$.
\end{prob}

{\bf Advertencia 2}: En general, el símbolo de Jacobi NO satisface la
ecuación~\eqref{eq-legendre}. Eso solo vale para $N$ primo impar.

\begin{prop}\label{prop-euler-jacobi}
Si $N\geq 3$ es un entero impar que es divisible por al menos dos
primos distintos, entonces
\[
\left|\{1\leq a<N\mid \gcd(a,N)=1\,\wedge\,a^{(N-1)/2}\equiv
\left(\frac{a}{N}\right)\pmod{N}\}\right|\leq\frac{\varphi(N)}2
\]
\end{prop}
\begin{proof}
Consideremos la aplicación $\eta:(\ZZ/N\ZZ)^*\rightarrow(\ZZ/N\ZZ)^*$
dado por $[a]\mapsto \left[a^{(N-1)/2}\left(\frac{a}{N}\right)\right]$.
Como el símbolo de Jacobi es multiplicativo, se tiene que $\eta$ es un
morfismo de grupos. Bastará con probar que ${\rm Im}(\eta)$ tiene al
menos dos elementos, ya que en ese caso tendríamos $|{\rm Ker}(\eta)|=
\frac{|(\ZZ/N\ZZ)^*|}{|{\rm Im}(\eta)|}\leq\frac{\varphi(N)}2$, tal
como afirma la proposición. Vamos a proceder por contradicción, es
decir, vamos a suponer que ${\rm Im}(\eta)=\{[1]\}$. En particular tenemos
que $a^{N-1}\equiv1\pmod{N}$ para todo $a$ coprimo con $N$, es decir,
$N$ es un número de Carmichael. Por el criterio de Korselt, $N$ debe ser
libre de cuadrados, es decir, $N=p_1p_2\cdots p_k$ para los $p_i$ primos
impares distintos tales que
$p_i-1\mid N-1$. Usando el teorema chino del resto, construimos un entero~$a$
que no sea un cuadrado módulo $p_1$ y $p_2$ y que sea $1$ módulo los
demás $p_i$. Por definición, se tiene que $\left(\frac{a}{p_1}\right)=
\left(\frac{a}{p_2}\right)=-1$ y $\left(\frac{a}{p_i}\right)=1$ para
$i\geq 3$. En particular, $\left(\frac{a}{N}\right)=1$ y
por lo tanto debe ser $a^{(N-1)/2}\equiv 1\pmod{N}$ de acuerdo con la
hipótesis ${\rm Im}(\eta)=\{[1]\}$. Esta última
congruencia implica que $a^{(N-1)/2}\equiv1\pmod{p_1}$. Ahora volvemos
a utilizar el teorema chino del resto para producir un entero $b$
que sea $a$ módulo $p_1$ y que sea $1$ módulo los demás~$p_i$.
Se tiene entonces que $\left(\frac{b}{N}\right)=-1$ y por lo tanto
que $b^{(N-1)/2}\equiv-1\pmod{N}$. Esto nos conduce a la contradicción
$b^{(N-1)/2}\equiv-1\pmod{p_1}$, ya que $a\equiv b\pmod{p_1}$ por
construcción.
\end{proof}

La proposición anterior suele usarse para comprobar rápidamente si
un número impar $N$ es primo o no. La idea es elegir un entero
$a\in\{1,\ldots,N-1\}$ uniformemente al azar y comprobar que $\gcd(a,N)=1$
y que $a^{(N-1)/2}\equiv\left(\frac{a}{N}\right)\pmod{N}$, utilizando
algoritmos rápidos para el máximo común divisor, para el símbolo de Jacobi
y para la exponenciación modular. Si $N$ no pasa el test, queda demostrado
que no es primo. En caso contrario, se vuelve a repetir $k$ veces la
prueba. De acuerdo con la proposición~\ref{prop-euler-jacobi}, la probabilidad
de que un número compuesto pase $k$ veces el test es a lo sumo $2^{-k}$.
Este método se llama de Solovay-Strassen.

\begin{prob}
Implementar en Python3 la función \texttt{test\_de\_solovay\_strassen(n,k)}
que aplique $k\geq1$ iteraciones del test de Solovay-Strassen al entero
$n\geq2$ dado. Si $n$ no pasa el test, la función debe devolver la cadena
\texttt{"compuesto"} y en caso contrario \texttt{"probablemente primo"}.
Los casos donde $n$ es par se deben resolver directamente (son triviales)
antes de aplicar los tests.
\end{prob}

\begin{prob}
Una versión efectiva del teorema del número primo afirma que
\[
\frac{x}{\ln(x)+2}<\pi(x)<\frac{x}{\ln(x)-4}\quad\forall\,x\geq 55,
\]
donde $\pi(x)$ es la cantidad de números primos en el intervalo $[1,x]$
y $\ln(x)$ es el logaritmo natural. En particular, la probabilidad de
encontrar un número primo en el intervalo $[1,10^{300}-1]$ eligiendo
uniformemente al azar $300$ dígitos decimales es aproximadamente
$1.45\cdot 10^{-3}$, es decir, uno entre 690. Escribir un programa
que genere números de $300$ dígitos decimales al azar (dígito a
dígito), que aplique el test de Solovay-Strassen con $k=20$ y que se
detenga al encontrar un entero que pase el test, es decir, uno que
sea \texttt{"probablemente primo"}. ¿Cuántos enteros fueron explorados
hasta conseguir el resultado? ¿Qué pasa si se repite el experimento
varias veces?
\end{prob}

Ya estamos en condiciones de entender cuáles ecuaciones $x^2\equiv a\pmod{N}$
son resolubles para cualquier $N\geq 2$ y $a\in\ZZ$. Lo primero
que debemos hacer es factorizar $N=p_1^{n_1}p_2^{n_2}\cdots p_k^{n_k}$ y de
ese modo reducir el problema al caso $N=p^n$ para $p$ primo y $n\geq 1$.
Por el teorema chino del resto, $x^2\equiv a\pmod{N}$ tiene solución
si y solo si $x^2\equiv a\pmod{p_i^{n_i}}$ tiene solución para todo
$i=1,\ldots,k$. El caso $n=1$ corresponde con todo lo que vimos antes y se
resuelve eficientemente utilizando el símbolo de Jacobi. El caso $n>1$ no
aporta demasiada dificultad, como muestra el siguiente resultado. Solo
discutimos el caso $\gcd(a,N)=1$, es decir, $p\nmid a$, ya que es fácil
reducirse siempre a esa situación: si $a=p^ub$ con $p\nmid b$ y $1\leq u<n$,
entonces $x^2\equiv a\pmod{N}$ tiene solución si y solo si $u$ es par y $b$
es resto cuadrático módulo $N$.

\begin{prop}\label{prop-88}
Sea $N=p^n$ con $p$ primo impar y $n\geq 1$. Sea $a\in\ZZ$ coprimo con $N$,
es decir, $p\nmid a$. Entonces $a$ es resto cuadrático módulo $N$ si y solo
si $a$ es resto cuadrático módulo $p$.
\end{prop}
\begin{proof}
Bastará con probar que si la ecuación $x^2\equiv a\pmod{p^k}$ tiene solución
para un cierto $k\geq 1$, entonces tambíen la tendrá la ecuación $x^2\equiv a
\pmod{p^{k+1}}$. Sea $x_0\in\ZZ$ tal que $x_0^2=a+p^kt$ para cierto $t\in\ZZ$,
es decir, $x_0$ es una solución de la congruencia $x^2\equiv a\pmod{p^k}$.
Tenemos que $(x_0+p^k\ell)^2=x_0^2+2p^k\ell x_0+p^{2k}\ell^2\equiv
a+p^k(t+2\ell x_0)\pmod{p^{k+1}}$. Eligiendo $\ell$ tal que
$-2 x_0\ell\equiv t\pmod{p}$ se consigue que $x_0+p^k\ell$ sea una solución
de $x^2\equiv a\pmod{p^{k+1}}$. La elección de $\ell$ es posible ya que
$\gcd(-2x_0,p)=1$.
\end{proof}

\begin{prob}
Sea $n\geq 3$ y sea $a\in\ZZ$. Demostrar con un argumento inductivo
que la congruencia $x^2\equiv a\pmod{2^n}$ tiene solución si y solo si
$a\equiv 1\pmod{8}$.
\end{prob}

\bigskip

Solo nos queda resolver el problema {\bf (SOL)}, es decir, sabiendo
que $x^2\equiv a\pmod{N}$ tiene solución para $N\geq 2$ y $a\in\ZZ$
coprimo con $N$ dados, obtener algorítmicamente los posibles $x$.
Por el teorema chino del resto podemos reducirnos al caso $N=p^n$
para $p$ primo y $n\geq 1$, y por el argumento inductivo visto en
la demostración de la proposición~\ref{prop-88} bastará con considerar
el caso $n=1$ y $p$ primo impar.

\bigskip

{\bf Algoritmo de Tonelli-Shanks}:
Sea $p$ un primo impar y supongamos que $p=r2^k+1$ con $k\geq 1$ y
$r$ impar. Como primera aproximación, calculamos $x=a^{(r+1)/2}\bmod{p}$, que
satisface $x^2\equiv a^{r+1}\equiv aa^r\pmod{p}$. Si $a^r\equiv 1\pmod{p}$,
hemos terminado, ya que tendríamos $x^2\equiv a\pmod{p}$. En caso
contrario, observemos que el valor $z=a^r\bmod{p}$ es una raíz
$(2^{k-1})$-ésima de la unidad módulo $p$. En efecto, se tiene que
$z^{2^{k-1}}\equiv a^{r2^{k-1}}\equiv a^{(p-1)/2}\equiv1\pmod{p}$, ya que~$a$
es un resto cuadrático. Es decir, tenemos que $x^2\equiv az\pmod{p}$ con
$z$ una raíz $(2^{k-1})$-ésima de la unidad. Iremos cambiando los valores
de $x$ y $z$ adecuadamente en esa expresión, para ir consiguiendo
iterativamente que $z$ sea una raíz $(2^{k-2})$-ésima de la unidad,
luego una raíz $(2^{k-3})$-ésima, y así siguiendo, hasta llegar a
$z=1$. En ese punto, el valor de $x$ es la raíz cuadrada buscada. Para
la iteración, supondremos que $x^2\equiv az\pmod{p}$ con $z$ una raíz
$(2^s)$-ésima de la unidad módulo $p$ para un cierto $1\leq s\leq k-1$.
Lógicamente, si resulta que $z$ es también una raíz $(2^{s-1})$-ésima,
no hay nada que hacer. Notar que esta condición puede testearse
fácilmente comprobando que $z^{2^{s-1}}\equiv 1\pmod{p}$. El caso
interesante es el otro, en el que se tiene necesariamente que
$z^{2^{s-1}}\equiv -1\pmod{p}$. El truco aquí es reemplazar
$x$ por $xt\pmod{p}$ y $z$ por $zt^2\pmod{p}$, donde $t$ satisface
$t^{2^s}\equiv -1\pmod{p}$. Una forma sencilla de conseguir un valor
de $t$ que cumpla con esa condición es tomar $t\equiv g^{(p-1)/2^{s+1}}\equiv
g^{r2^{k-s-1}}\pmod{p}$ para una raíz primitiva cualquiera~$g$ módulo~$p$.

\begin{prob}
Implementar la función \texttt{raiz\_cuadrada\_modular(a,p)} que calcule
una solución de la congruencia $x^2\equiv a\pmod{p}$ para $a$ un resto
cuadrático módulo $p$ y $p$ un primo impar. Para conseguir una raíz primitiva
módulo $p$ se podrá utilizar la función \texttt{menor\_raiz\_primitiva(p)}
del problema~4.7.
\end{prob}

\begin{prob}
Encontrar TODAS las soluciones de la ecuación
$x^2\equiv -25\pmod{5^3\cdot 13^4\cdot 17}$. Usar el teorema chino
del resto para reducir el problema a módulos potencias de primo y
el argumento de la demostración de la proposición~\ref{prop-88} para,
por ejemplo, producir una solución de $x^2\equiv-25\pmod{13^4}$ a
partir de una de $x^2\equiv-25\pmod{13}$.
\end{prob}

\end{document}