\documentclass[a4paper, 11pt]{article}

\usepackage[utf8]{inputenc}
\usepackage{amsmath,amssymb,amsthm}
\usepackage[margin=2cm]{geometry}
\usepackage{listings}
\usepackage{color}
\usepackage{xurl}
\usepackage{graphicx}

\definecolor{deepblue}{rgb}{0,0,0.5}
\definecolor{deepred}{rgb}{0.6,0,0}
\definecolor{deepgreen}{rgb}{0,0.5,0}
\definecolor{lightgrey}{rgb}{0.95,0.95,0.95}

\renewcommand*{\lstlistingname}{Programa}

\lstset{
   language=Python,
   basicstyle=\ttfamily\small,
   keywordstyle=\color{deepblue}\bfseries\itshape,
   commentstyle=\color{deepgreen}\itshape,
   stringstyle=\color{deepred},
   backgroundcolor=\color{lightgrey},
   morekeywords={as,assert,nonlocal,with,yield},
   showstringspaces=false,
   numbers=left,
   rulecolor=\color{black},
   captionpos=b,
   frame=leftline,
   literate=
   {á}{{\'a}}1 {é}{{\'e}}1 {í}{{\'i}}1 {ó}{{\'o}}1 {ú}{{\'u}}1
   {Á}{{\'A}}1 {É}{{\'E}}1 {Í}{{\'I}}1 {Ó}{{\'O}}1 {Ú}{{\'U}}1
   {à}{{\`a}}1 {è}{{\`e}}1 {ì}{{\`i}}1 {ò}{{\`o}}1 {ù}{{\`u}}1
   {À}{{\`A}}1 {È}{{\`E}}1 {Ì}{{\`I}}1 {Ò}{{\`O}}1 {Ù}{{\`U}}1
   {ä}{{\"a}}1 {ë}{{\"e}}1 {ï}{{\"i}}1 {ö}{{\"o}}1 {ü}{{\"u}}1
   {Ä}{{\"A}}1 {Ë}{{\"E}}1 {Ï}{{\"I}}1 {Ö}{{\"O}}1 {Ü}{{\"U}}1
   {â}{{\^a}}1 {ê}{{\^e}}1 {î}{{\^i}}1 {ô}{{\^o}}1 {û}{{\^u}}1
   {Â}{{\^A}}1 {Ê}{{\^E}}1 {Î}{{\^I}}1 {Ô}{{\^O}}1 {Û}{{\^U}}1
   {ã}{{\~a}}1 {ẽ}{{\~e}}1 {ĩ}{{\~i}}1 {õ}{{\~o}}1 {ũ}{{\~u}}1
   {Ã}{{\~A}}1 {Ẽ}{{\~E}}1 {Ĩ}{{\~I}}1 {Õ}{{\~O}}1 {Ũ}{{\~U}}1
   {œ}{{\oe}}1 {Œ}{{\OE}}1 {æ}{{\ae}}1 {Æ}{{\AE}}1 {ß}{{\ss}}1
   {ű}{{\H{u}}}1 {Ű}{{\H{U}}}1 {ő}{{\H{o}}}1 {Ő}{{\H{O}}}1
   {ç}{{\c c}}1 {Ç}{{\c C}}1 {ø}{{\o}}1 {Ø}{{\O}}1 {å}{{\r a}}1 {Å}{{\r A}}1
   {€}{{\euro}}1 {£}{{\pounds}}1 {«}{{\guillemotleft}}1
   {»}{{\guillemotright}}1 {ñ}{{\~n}}1 {Ñ}{{\~N}}1 {¿}{{?`}}1 {¡}{{!`}}1 
}

\newcommand{\NN}{\mathbb{N}}
\newcommand{\ZZ}{\mathbb{Z}}
\newcommand{\QQ}{\mathbb{Q}}
\newcommand{\RR}{\mathbb{R}}
\newcommand{\CC}{\mathbb{C}}
\newcommand{\FF}{\mathbb{F}}
\newcommand{\PP}{\mathbb{P}}
\newcommand{\UU}{\mathcal{U}}
\newcommand{\XX}{\mathcal{X}}
\newcommand{\sk}{\textrm{sk}}
\newcommand{\SK}{\textrm{SK}}
\newcommand{\Enc}{\textrm{Enc}}
\newcommand{\Dec}{\textrm{Dec}}

\newcounter{numerodetema}
\newcommand\tema[1]{{\eject\stepcounter{numerodetema}\bf Álgebra Computacional
2023/24 \hfill Tema~\#\arabic{numerodetema}}\par\smallskip\hrule\bigskip\par
\begin{center}{\Large\bf #1}\end{center}}

\theoremstyle{plain}
\newtheorem{teor}{Teorema}[numerodetema]
\newtheorem{coro}[teor]{Corolario}
\newtheorem{lema}[teor]{Lema}
\newtheorem{prop}[teor]{Proposición}
\theoremstyle{definition}
\newtheorem{defi}[teor]{Definición}
\newtheorem{prob}{Problema}[numerodetema]

\newcommand\myatop[2]{\genfrac{}{}{0pt}{}{#1}{#2}}

\parindent=0cm

\begin{document}

\setcounter{numerodetema}{7}
\tema{Curvas elípticas}

\bigskip

En esta parte trabajaremos con un cuerpo $K$ y denotaremos $K^2$ al plano afín y
$\PP^2(K)$ al plano proyectivo. El cuerpo $\overline{K}$ es la clausura algebraica
de $K$. Al conjunto de ceros de un polinomio homogéneo
$F\in K[x,y,z]$ no nulo
\[
   \mathcal{C}_F=\left\{[x_0:y_0:z_0]\in\PP^2(\overline{K})\,:\,F(x_0,y_0,z_0)=0\right\}
\]
se lo llama \emph{curva algebraica plana proyectiva definida sobre $K$}. Por
conveniencia denotaremos $\mathcal{C}_F(K)$ a los puntos de la curva con coordenadas
en $K$, es decir, $\mathcal{C}_F\cap\PP^2(K)$.

\bigskip

\begin{lema}\label{lema-int-finita}
Consideremos una curva $\mathcal{C}_F$ dada por un polinomio homogéneo (no nulo)
$F\in\overline{K}[x,y,z]$ de grado $d\geq 0$ y sea $L(x,y,z)=ax+by+cz\in
\overline{K}[x,y,z]$ un polinomio lineal (no nulo) que no divide a $F(x,y,z)$.
Entonces
$|\mathcal{C}_F\cap\mathcal{C}_L|\leq d$.
\end{lema}
\begin{proof}
Supongamos sin pérdida de generalidad que $a\neq 0$. Sustituyendo $x=-\frac{by+cz}{a}$ en
$F(x,y,z)=0$ obtenemos un polinomio homogéneo $G(y,z)\in\overline{K}[y,z]$ cuyos ceros
$[y_0:z_0]$ están en correspondencia uno a uno con los puntos de la intersección
$\mathcal{C}_F\cap\mathcal{C}_L$. Por construcción, se tiene que $F\equiv G\pmod{L}$
en $\overline{K}[x,y,z]$. Como $L\nmid F$ y $L$ es irreducible (por ser de grado $1$),
se sigue que $L\nmid G$ y en particular, $G$ es un polinomio no nulo. Además, por
construcción, $G$ es homogéneo de grado $d$. Cada cero $[y_0:z_0]$ de $G$ corresponde
con el factor lineal $y_0z-z_0y$, o un asociado, en la factorización de $G$. Esto prueba
que no puede haber mas de $d$ ceros.
\end{proof}

El lema~\ref{lema-int-finita} nos da una idea de como definir la multiplicidad de
intersección ${\rm mult}_P(L,F)$ de una función lineal no nula $L=ax+by+cz\in
\overline{K}[x,y,z]$ y un polinomio homogéneo no nulo $F\in\overline{K}[x,y,z]$
de grado $d$ en un punto común $P=[x_0:y_0:z_0]\in\mathcal{C}_L\cap\mathcal{C}_F$.
Suponiendo que $a\neq 0$ y que $L\nmid F$, podemos hacer la sustitución $G(y,z)=
F(-\frac{by+cz}{a},y,z)$ como en la demostración del lema y definir la multiplicidad
de intersección como el orden con el que factor $y_0z-z_0y$ aparece en $G(y,z)$.
\[
    {\rm mult}_P(L,F)=\max\left\{n\geq 0\,:\,(y_0z-z_0y)^n\mid F\left(-\frac{by+cz}{a},y,z\right)\right\}\qquad\text{si}\;a\neq0\;\text{y}\;L\nmid F
\]
De ese modo tenemos que la suma de las  multiplicidades de intersección de todos
los puntos comunes es \emph{exactamente} $d$ por estar trabajando en el cuerpo
algebraicamente cerrado $\overline{K}$.
\begin{equation}\label{eq-mult}
    \sum_{P\in\mathcal{C}_L\cap\mathcal{C}_F}{\rm mult}_P(L,F)=\deg(G)=\deg(F)=d
\end{equation}
Por coherencia, deberíamos comprobar que si la función lineal tiene también $b\neq 0$
y hacemos la sustitución $H(x,z)=F(x,-\frac{ax+cz}{b},z)$, el orden con el que el factor
$z_0x-x_0z$ aparece en $H$ coincide con el que definimos antes.
\[
\begin{aligned}
    & (z_0x-x_0z)^n\mid H(x,z) \\
\Longleftrightarrow\quad
    & (z_0x-x_0z)^n\mid F\left(x,-\frac{ax+cz}{b}, z\right) \\
\Longleftrightarrow\quad
    & \left(z_0(-\frac{by+cz}{a})-x_0z\right)^n\mid F\left(-\frac{by+cz}{a}, y, z\right) \\
\Longleftrightarrow\quad
    & (-bz_0y-cz_0z-ax_0z)^n\mid  G(y,z) \\
\Longleftrightarrow\quad
    & (-bz_0y+by_0z)^n\mid  G(y,z) \\
\Longleftrightarrow\quad
    & (z_0y-y_0z)^n\mid  G(y,z)
\end{aligned}
\]
La segunda implicación se sigue de hacer el cambio de variables $x=-\frac{by+cz}a$ y la
penúltima de que el punto $[x_0:y_0:z_0]$ satisface la ecuación de la recta $L$.
Con esta comprobación vemos que en realidad podemos ``despejar'' cualquiera de las
variables de la función lineal (siempre que su coeficiente sea no nulo), sustituirlo
en $F(x,y,z)$ y luego mirar en el polinomio que queda de dos variables cuales son las
multiplicidades.
\[
\begin{aligned}
    {\rm mult}_P(L,F) &=\max\left\{n\geq 0\,:\,(z_0x-x_0z)^n\mid F\left(x,-\frac{ax+cz}{b}, z\right)\right\}\qquad\text{si}\;b\neq0\;\text{y}\;L\nmid F \\
    {\rm mult}_P(L,F) &=\max\left\{n\geq 0\,:\,(y_0x-x_0y)^n\mid F\left(x,y,-\frac{ax+by}{c}\right)\right\}\qquad\text{si}\;c\neq0\;\text{y}\;L\nmid F \\
\end{aligned}
\]
\bigskip

Veamos ahora otra forma de calcular la multiplicidad de interescción. Supongamos
nuevamente $a\neq 0$ y sea $P=[x_0:y_0:z_0]\in\mathcal{C}_L\cap\mathcal{C}_F$. No
puede ser $y_0=z_0=0$, ya que en ese caso tendríamos $x_0=0$ por la ecuación de la recta,
lo que no define ningún punto. Supongamos que $z_0\neq 0$. Podemos parametrizar la recta
$L$ en el plano afín $z=z_0$
\[
\begin{aligned}
 x & = x_0 + \frac{bt}{a}\\
 y & = y_0 - t \\
 z & = z_0
\end{aligned}
\]
utilizando el parámetro $t$. Si sustituimos esto en el polinomio $F$, nos queda
\[
  R(t)=F\left(x_0+\frac{bt}{a},y_0-t,z_0\right)\in\overline{K}[t]
\]
y la multiplicidad de intersección ${\rm mult}_P(L,F)$ será la mayor potencia de $t$
que divide a $R(t)$. Tenemos que comprobar que esta definición coincide con la que
vimos antes.
\[
\begin{aligned}
 & t^n\mid R(t) \\
\Longleftrightarrow\quad
 & t^n\mid F\left(x_0+\frac{bt}{a},y_0-t,z_0\right) \\
\Longleftrightarrow\quad
 & t^n\mid F\left(\frac{x_0z}{z_0}+\frac{bt}{a},\frac{y_0z}{z_0}-t,z\right) \\
\Longleftrightarrow\quad
 & \left(\frac{y_0z}{z_0}-y\right)^n\mid
    F\left(\frac{x_0z}{z_0}+\frac{b}{a}\left(\frac{y_0z}{z_0}-y\right),y,z\right) \\
\Longleftrightarrow\quad
 & (y_0z-z_0y)^n\mid F\left(\frac{ax_0z+by_0z-az_0y}{bz_0},y,z\right) \\
\Longleftrightarrow\quad
 & (y_0z-z_0y)^n\mid F\left(\frac{-cz-ay}{b},y,z\right) \\
\Longleftrightarrow\quad
 & (y_0z-z_0y)^n\mid G(y,z) \\
\end{aligned}
\]
La segunda implicación se obtiene homogeneizando, la siguiente haciendo el
cambio de variables $t=y_0z/z_0-y$ y la penúltima utilizando que $P$ satisface
la ecuación de la recta.

\bigskip

Hemos sido cuidadosos de hablar siempre de la multiplicidad de intersección de
los polinomios $L$ y $F$ y no de las curvas $\mathcal{C}_L$ y $\mathcal{C}_F$.
Si bien esta diferencia de lenguaje no afecta en absoluto a lo de la recta,
puede haber ambigüedades en el caso de la curva. El problema proviene de que
puede haber muchos polinomios que definan exactamente la misma curva. Por
ejemplo, si consideramos $F_1=(xz-y^2+z^2)^3(x^3+3xz^2)^8\in\CC[x,y,z]$ y
$F_2=(xz-y^2+z^2)(x^3+3yz^2)\in\CC[x,y,z]$, ambos definen la misma curva
$\mathcal{C}_{F_1}=\mathcal{C}_{F_2}$. Si quitamos las potencias de los factores,
este problema desaparece.

\begin{teor}\label{teor-biyeccion}
Hay una correspondencia biyectiva entre curvas algebraicas planas proyectiva y
polinomios de $\overline{K}[x,y,z]$ que no tienen factores repetidos (libres de
cuadrados) salvo constantes multiplicativas.
\end{teor}

Gracias al teorema~\ref{teor-biyeccion} tiene sentido ahora hablar tanto del grado
de una curva algebraica plana proyectiva como de la multiplicidad de intersección
de una recta y una curva. En ambos casos se utiliza el (único) polinomio libre de
cuadrados que define la curva.

\bigskip

Si una recta corta a una curva con multiplicidad mayor o igual que $2$, se dice
que es \emph{tangente} a la curva. Un punto de la curva por el que solo pasa una
recta tangente se dice \emph{simple} y en caso contrario se dice \emph{singular}.

\bigskip

Consideremos una curva $\mathcal{C}_F$ con $F$ libre de cuadrados y un punto
$P=[x_0:y_0:z_0]$ en la curva. Vamos a ver que si $\nabla F(P)\neq (0,0,0)$,
entonces la recta
\[
  \frac{\partial F}{\partial x}(P)\cdot x +
  \frac{\partial F}{\partial y}(P)\cdot y +
  \frac{\partial F}{\partial z}(P)\cdot z = 0
\]
pasa por $P$ y es la única tangente a la curva. Lo primero se deduce de la identidad
de Euler
\[
  \frac{\partial F}{\partial x}\cdot x +
  \frac{\partial F}{\partial y}\cdot y +
  \frac{\partial F}{\partial z}\cdot z = \deg(F)\cdot F
\]
sustituyendo el punto $P$. Para ver que esta recta es tangente, supongamos que
$a=\frac{\partial F}{\partial x}(P)\neq 0$ y $z_0\neq 0$. Tenemos que comprobar
que $t^2$ divide a $F(x_0+bt/a, y_0-t, z_0)$, donde $b=\frac{\partial F}{\partial y}(P)$.
Desarrollando en potencias de $t$, tenemos que
\[
F\left(x_0+\frac{bt}a, y_0-t, z_0\right)=F(P)+\left(\frac{b}{a}\frac{\partial F}{\partial x}(P)-\frac{\partial F}{\partial y}(P)\right)\cdot t + \text{términos con}\;t^2 +\cdots
\]
en donde se ve claramente que el término independiente y el lineal se anulan. Por otra
parte, si $ax+by+cz=0$ es una tangente cualquiera en $P$ (con $a\neq 0$), repetimos el
razonamiento y llegamos a que $a=\frac{\partial F}{\partial x}(P)$ (sin pérdida de
generalidad) y $b=\frac{\partial F}{\partial y}(P)$, ya que debe ser
\[
\frac{b}{a}\frac{\partial F}{\partial x}(P)-\frac{\partial F}{\partial y}(P) = 0.
\]
Utilizando que $ax_0+by_0+cz_0=0$ y la identidad de Euler, nos queda que
también será
\[
  c = -\frac{ax_0+by_0}{z_0} = -\frac{\frac{\partial F}{\partial x}(P)\cdot x_0 +
  \frac{\partial F}{\partial y}(P)\cdot y_0}{z_0} = \frac{\partial F}{\partial z}(P)
\]
lo que demuestra la unicidad de la recta tangente.

\begin{teor}
Sea $F\in\overline{K}[x,y,z]$ un polinomio homogéneo no nulo libre de cuadrados.
Un punto $P\in\mathcal{C}_F$ es simple si y solo si $\nabla F(P)\neq (0,0,0)$. En
este caso, la ecuación de la única recta tangente es $\nabla F(P)\cdot (x,y,z)=0$.
\end{teor}

\bigskip

Vamos a necesitar algunos resultados básicos de geometría algebraica. En una primera
lectura, se pueden omitir las demostraciones, ya que son bastante complicadas.

\begin{teor}\label{teor-1}
Sean $F,G\in K[x,y,z]$ polinomios homogéneos mónicos en $x$ de grados $d=\deg(F)\geq 1$
y $e=\deg(G)\geq 1$.
Son equivalentes:
\begin{enumerate}
\item Existe $H\in K[x,y,z]$ homogéneo de grado $\deg(H)\geq 1$ tal que $H\mid F$
y $H\mid G$.
\item Existen $A,B\in K[x,y,z]$ homogéneos no nulos tales que $AF+BG=0$, $\deg_x(A)\leq e-1$
y $\deg_x(B)\leq d-1$.
\item El determinante de la matriz
\[
\left[
\begin{array}{ccccccc}
F_0(y,z) & F_1(y,z) & \cdots   & F_d(y,z) &          &        &          \\
         & F_0(y,z) & F_1(y,z) & \cdots   & F_d(y,z) &        &          \\
         &          & \ddots   &          &          & \ddots &          \\
         &          &          & F_0(y,z) & F_1(y,z) & \cdots & F_d(y,z) \\
G_0(y,z) & G_1(y,z) & \cdots   & \cdots   & G_e(y,z) &        &          \\
         & \ddots   &          &          &          & \ddots &          \\
         &          & G_0(y,z) & G_1(y,z) & \cdots   & \cdots & G_e(y,z) \\
\end{array}
\right]\in K[y,z]^{(d+e)\times(d+e)}
\]
es cero.
\end{enumerate}
\end{teor}
\begin{proof}
$(1\Rightarrow 2)$: Como los polinomios $F$ y $G$ son mónicos en $x$, cualquier
factor común $H$ tendrá $\deg_x(H)\geq 1$. Basta tomar $A=G/H$ y $B=-F/H$.

$(2\Rightarrow 1)$: Supongamos que no existe $H$. Esto significa que $F$ y $G$ no
tienen ningún factor común en $K[x,y,z]$. Como $F\mid BG$ y $K[x,y,z]$ es un dominio
de factorización única, debe ser $F\mid B$. Esto es una contradicción, ya que
$\deg_x(B)<\deg_x(F)=\deg(F)$.

$(2\Rightarrow 3)$: El sistema de ecuaciones $AF+BG=0$ tiene solución (no nula) en
$K[y,z]$ donde las incógnitas son los $e$ coeficientes de $A$ y los $d$ coeficientes
de $B$. Como hay $d+e$ incógnitas y $d+e$ ecuaciones y el sistema es homogéneo,
se sigue que el determinante del sistema es cero. La matriz de este sistema es la
traspuesta de la matriz de arriba.

$(3\Rightarrow 2)$: Analizando el mismo sistema de ecuaciones, sabemos que debe
existir una solución no nula en $K(y,z)$ para los coeficientes de $A$ y $B$, ya
que el determinante del sistema homogéneo es cero. Limpiando denominadores (lo que
solo requiere multiplicar por polinomios en $K[y,z]$), obtenemos $A,B\in K[x,y,z]$
con $\deg_x(A)\leq e-1$ y $\deg_x(B)\leq d-1$.
\end{proof}

Al determinante de la matriz de arriba se lo llama \emph{resultante de $F$
y $G$ con respecto a la variable $x$} y se lo escribe ${\rm Res}_x(F,G)$.

\begin{teor}\label{teor-2}
Sean $F,G\in K[x,y,z]$ polinomios homogéneos mónicos en $x$ de grados $d=\deg(F)\geq 1$ y
$e=\deg(G)\geq 1$. Entonces ${\rm Res}_x(F,G)\in K[y,z]$ es cero o un polinomio
homogéneo de grado $de$ y existen $A,B\in K[x,y,z]$ homogéneos de grados
$\deg_x(A)\leq e-1$ y $\deg_x(B)\leq d-1$ tales que ${\rm Res}_x(F,G)=AF+BG$.
\end{teor}
\begin{proof}
El determinante de la matriz $M$ es una suma de $(d+e)!$ productos de la forma
\[M_{1,\sigma(1)}M_{2,\sigma(2)}\cdots M_{d+e,\sigma(d+e)}\] donde $\sigma$ es
una permutación de las $d+e$ columnas. Cada una de las entradas es cero o un
polinomio homogéneo. Veremos que cada sumando no nulo es siempre de grado $de$.
En efecto, $\deg(M_{i,\sigma(i)})=d-i+\sigma(i)$ para $i=1,\ldots,e$ y
$\deg(M_{i,\sigma(i)})=e+1-i+\sigma(i)$ para $i=e+1,\ldots,e+d$.
Sumando para $i=1,\ldots,d+e$ nos queda grado $de$. Para la otra parte, supondremos
que ${\rm Res}_x(F,G)\neq 0$, ya que el caso cero lo analizamos en el teorema anterior.
Aplicamos la regla de Cramer al sistema de ecuaciones $AF+BG=1$ con incógnitas los
coeficientes de $A$ y $B$ en $K(y,z)$. El único denominador que aparece es el
determinante del sistema ${\rm Res}_x(F,G)$. Limpiando ese denominador, obtenemos
a la resultante como combinación lineal (con coeficientes en $K[x,y,z]$) de $F$ y
$G$. En caso de que $A$ y $B$ no sean homogéneos, podemos simplemente extraer sus
partes homogéneas de grado $d(e-1)$ y $(d-1)e$.
\end{proof}

\begin{teor}[Bezout]
Sean $F,G\in \overline{K}[x,y,z]$ polinomios homogéneos no nulos de grados
$d=\deg(F)\geq 1$ y $e=\deg(G)\geq 1$. Entonces sucede una y solo una de las
siguientes:
\begin{enumerate}
\item $F$ y $G$ tienen un factor común $H\in\overline{K}[x,y,z]$ de grado $\deg(H)\geq 1$
\item $\left|\mathcal{C}_F\cap\mathcal{C}_G\right|\leq de$.
\end{enumerate}
\end{teor}
\begin{proof}
Aplicando un cambio lineal de variables genérico, podemos suponer sin pérdida de
generalidad que $F$ y $G$ son mónicos en $x$. Si no sucede (1), el teorema~\ref{teor-1}
nos dice que ${\rm Res}_x(F,G)$ es un polinomio no nulo y el teorema~\ref{teor-2},
nos dice que es homogéneo de grado $de$. Cada punto $[x:y:z]\in\mathcal{C}_F
\cap\mathcal{C}_G$ nos da un cero $[y:z]$ de la resultante, ya que esta
es combinación lineal de $F$ y $G$. En particular, hay a lo sumo $de$ posibles $[y:z]$.
Sustituyendo estos en $F$, obtenemos un polinomio mónico en $x$ y por lo tanto
finitos posibles valores de $x$ en $\overline{K}$ que satisfacen esa ecuación. Esto
implica que la intersección de las curvas es finita. Nuevamente haciendo un cambio
lineal genérico de coordenadas, podemos suponer sin pérdida de generalidad que todos
los puntos de la intersección tienen $[y:z]$ distintos, lo que prueba (2). Solo
resta ver que si sucede (1), no puede suceder (2). Esto es inmediato ya que cualquier
polinomio $H$ de grado $\deg(H)\geq 1$ tiene infinitas soluciones en el cuerpo
$\overline{K}$ y además $\mathcal{C}_H\subseteq\mathcal{C}_F\cap\mathcal{C}_G$.
\end{proof}

\begin{proof}[Proof of theorem~\ref{teor-biyeccion}]
En primer lugar, observemos que cualquier curva puede ser definida por un polinomio
libre de cuadrados en $\overline{K}[x,y,z]$. Lo que resta ver es que si tenemos dos
polinomios $F,G\in\overline{K}[x,y,z]$ libres de cuadrados distintos, entonces
$\mathcal{C}_F\neq\mathcal{C}_G$. Sea $H$ un factor irreducible de $F$ que no esté
en $G$. Aplicando el teorema de Bezout, sabemos que $\mathcal{C}_H\cap\mathcal{C}_G$
es un conjunto finito. Como $\mathcal{C}_H$ es infinito, podemos elegir $P\in\mathcal{C}_H\setminus\mathcal{C}_G$. Ese punto también estará en $\mathcal{C}_F$, ya que $H\mid F$.
Esto prueba que en $\mathcal{C}_F$ hay al menos un punto que no está en $\mathcal{C}_G$,
tal como afirmábamos.
\end{proof}

\begin{teor}\label{int-no-vacia}
Sea $F,G\in\overline{K}[x,y,z]$ polinomios homogéneos no nulos de grados $d=\deg(F)\geq1$
y $e=\deg(G)\geq 1$. Entonces $\mathcal{C}_F\cap\mathcal{C}_G\neq\emptyset$.
\end{teor}
\begin{proof}
A través de un cambio genérico lineal de coordenadas, podemos suponer sin pérdida de
generalidad que $F$ y $G$ son mónicos en $x$.
En el caso de que $F$ y $G$ tengan un factor común $H$, esto es inmediato, ya que
cualquier punto $P\in\mathcal{C}_H(\overline{K})$ cumple lo pedido (y hay infinitos de
estos). En el otro caso, sabemos que ${\rm Res}_x(F,G)=AF+BG\in\overline{K}[y,z]$ es un
polinomio homogéneo no nulo de grado $de\geq 1$ (por los teoremas~\ref{teor-1}
y~\ref{teor-2}) para ciertos $A,B\in\overline{K}[x,y,z]$. Como estamos trabajando sobre
el cuerpo algebraicamente cerrado $\overline{K}$, existirá un $[y_0:z_0]$ que anule a
la resultante. Los polinomios $F(x,y_0w,z_0w), G(x,y_0w,z_0w)\in\overline{K}[x,w]$
homogéneos (mónicos en $x$) satisfacen
\[
 A(x,y_0w,z_0w)F(x,y_0w,z_0w)+B(x,y_0w,z_0w)G(x,y_0w,z_0w)={\rm Res}_x(F,G)(y_0w,z_0w)=0
\]
y por lo tanto tienen un factor común en $\overline{K}[x,w]$ (con el mismo razonamiento
que en la implicación $2\Rightarrow1$ del teorema~\ref{teor-1}). Tomando un cero
$[x_0:w_0]$ de ese factor común, obtenemos el punto $[x_0:y_0w_0:z_0w_0]$ que anula a $F$ y
$G$.
\end{proof}

\bigskip

\begin{defi}\label{def-elip}
Una \emph{curva elíptica definida sobre $K$} es una curva proyectiva plana
$\mathcal{C}_F$ para un cierto $F\in K[x,y,z]$ homogéneo no nulo de grado $3$
tal que:
\begin{enumerate}
\item $\mathcal{C}_F(K)\neq\emptyset$.
\item Para todo $P\in\mathcal{C}_F$ se tiene
\[
   \nabla F(P)=\left(\frac{\partial F}{\partial x}(P),
                     \frac{\partial F}{\partial y}(P),
                     \frac{\partial F}{\partial z}(P)\right)\neq (0,0,0),
\]
es decir, $\mathcal{C}_F$ es una curva \emph{lisa} (sin puntos singulares).
\end{enumerate}
\end{defi}

\bigskip

Una curva elíptica $\mathcal{C}_F$ siempre está dada por un polinomio irreducible
en $\overline{K}[x,y,z]$. En efecto, si el polinomio $F$ se pudiera factorizar como
$F=GH$ para ciertos $G,H\in\overline{K}[x,y,z]$ homogéneos (de grados $\geq 1$),
podríamos tomar un punto $P$ tal que $G(P)=H(P)=0$ (por el teorema~\ref{int-no-vacia})
y comprobar que
\[
   \nabla F(P)=\nabla(GH)(P)=G(P)\cdot\nabla H(P) + H(P)\cdot\nabla G(P) = (0,0,0)
\]
lo que contradice la propiedad (2) de la definición~\ref{def-elip}.

\bigskip

Recordemos que una recta corta a una curva elíptica en exactamente $3$ puntos (contados
con multiplicidades).

\begin{defi}
Sea $\mathcal{C}_F$ una curva elíptica definida sobre $K$. Sean $P,Q\in\mathcal{C}_F(K)$
puntos de la curva (pueden ser el mismo). Entonces la recta que pasa por $P$ y $Q$ (la
tangente en caso de ser $P=Q$) corta a la curva en un tercer punto que se denota $P*Q\in\mathcal{C}_F(K)$ (que también podría ser el mismo).
\end{defi}

Para que la definición anterior tenga sentido, habría que comprobar que el punto $P*Q$
está en $\mathcal{C}_F(K)$. Para esto, basta ver que la recta que pasa por los dos
puntos $P$ y $Q$ (o la tangente si $P=Q$) tiene coeficientes en $K$ y que si un polinomio
homogéneo de grado $3$ en $K[y,z]$ tiene dos factores en $K[y,z]$, el tercer factor
también lo estará.

\bigskip

\begin{defi}
Sea $\mathcal{C}_F$ una curva elíptica definida sobre $K$ y sea $\mathcal{O}
\in\mathcal{C}_F(K)$ un punto fijo de la curva. La operación $+:\mathcal{C}_F(K)
\times\mathcal{C}_F(K)\to\mathcal{C}_F(K)$ se define como $P+Q=\mathcal{O}*(P*Q)$.
\end{defi}

\begin{teor}\label{teor-suma}
Sea $\mathcal{C}_F$ una curva elíptica definida sobre $K$ y sea $\mathcal{O}
\in\mathcal{C}_F(K)$ un punto fijo de la curva. Entonces $(\mathcal{C}_F(K),+)$ es
un grupo abeliano con neutro $\mathcal{O}$.
\end{teor}

Antes de demostrar nuestro resultado principal sobre curvas elípticas, necesitaremos
dos resultados importantes.

\begin{prop}
Sean $A_1,\ldots,A_8\in\PP^2(\overline{K})$ tales que no hay $4$ alineados y no hay
$7$ en una cónica (una curva algebraica proyectiva de grado $2$). Entonces el
$\overline{K}$-espacio vectorial
\[
\mathbb{S}=\left\{F\in\overline{K}[x,y,z]\;\text{homogéneo de grado}\;3\,:\,
F(A_1)=\cdots=F(A_8)=0\right\}
\]
tiene dimensión $2$.
\end{prop}
\begin{proof}
El espacio de los polinomios homogéneos de grado $3$ tiene dimensión $10$, ya que
hay $10$ coeficientes en esos polinomios. Cada ecuación $F(A_i)=0$ es lineal en los
coeficientes. Como hay $8$ ecuaciones, tenemos que $\dim(\mathbb{S})\geq 2$. Lo que
resta ver es que estas ecuaciones son linealmente independientes. Para eso, bastará
con demostrar que siempre hay un polinomio $F\in\overline{K}[x,y,z]$ tal que
$F(A_1)=\cdots=F(A_7)=0$ pero $F(A_8)\neq0$. Consideramos los siguientes polinomios:
\[
\begin{aligned}
F_1 &= Q_{A_1A_2A_3A_4A_5}\cdot L_{A_6A_7} \\
F_2 &= Q_{A_1A_2A_3A_4A_6}\cdot L_{A_5A_7} \\
F_3 &= Q_{A_1A_2A_3A_4A_7}\cdot L_{A_5A_6} \\
\end{aligned}
\]
donde $Q_{ABCDE}$ denota un polinomio homogéneo no nulo de grado $2$ (cónica) que se
anula en $ABCDE$ y $L_{AB}$ es un polinomio homogéneo no nulo de grado $1$ (recta) que
se anula en $AB$. La existencia de las cónicas está garantizada porque hay $6$ coeficientes
y solo $5$ restricciones lineales. Así hemos producido tres polinomios homogéneos $F_1$,
$F_2$ y $F_3$ de grado $3$ no nulos. Veamos que alguno de estos no se anulará en $A_8$.
Procederemos por contradicción, es decir, supondremos que $F_1(A_8)=F_2(A_8)=F_3(A_8)=0$.
Sabemos que el punto $A_8$ no puede estar en dos de las rectas $L_{A_6A_7}$, $L_{A_5A_7}$,
$L_{A_5A_6}$ ya que en caso contrario habría $4$ puntos alineados. Esto demuestra que
$A_8$ debe estar en dos de las cónicas, que sin pérdida de generalidad supondremos que
$Q_{A_1A_2A_3A_4A_5}(A_8)=0$ y $Q_{A_1A_2A_3A_4A_6}(A_8)=0$. Si estas dos cónicas no
comparten un factor común, el teorema de Bezout nos dice que se cortarían en a lo sumo
$4$ puntos, pero estas tienen en común a $A_1A_2A_3A_4A_8$. Entonces deben tener un
factor común. La posibilidad de que ambas cónicas coincidan queda descartada puesto
que si no tendríamos los $7$ puntos $A_1A_2A_3A_4A_5A_6A_8$ en una cónica. La única
posibilidad que queda es que el factor común sea de grado $1$, es decir,
\[
\begin{aligned}
Q_{A_1A_2A_3A_4A_5A_8} &= L_1\cdot L_2 \\
Q_{A_1A_2A_3A_4A_6A_8} &= L_1\cdot L_3
\end{aligned}
\]
para ciertas rectas $L_1,L_2,L_3\in\overline{K}[x,y,z]$ y la recta $L_2$ distinta de
la $L_3$. Como estas cónicas tienen $5$ puntos en común $A_1A_2A_3A_4A_8$ y las rectas
$L_2$ y $L_3$ se cortan en a lo sumo un punto, tendrá que haber al menos $4$ de esos
puntos en $L_1$. Contradicción.
\end{proof}

\begin{teor}[Cayley-Bacharach]
Sean $F,G,H\in\overline{K}[x,y,z]$ polinomios homogéneos de grado $3$ no nulos con $F$
irreducible. Supongamos que $\mathcal{C}_G$ y $\mathcal{C}_H$ se intersectan en $9$
puntos \emph{distintos} $A_1,A_2,\ldots,A_9$ y que $F$ se anula en $A_1,\ldots,A_8$.
Entonces $F$ también se anulará en $A_9$.
\end{teor}
\begin{proof}
Empecemos por observar que los puntos $A_1,\ldots,A_8$ satisfacen las hipótesis de la
proposición anterior. Por Bezout cualquier recta cortará a $\mathcal{C}_F$ en a lo
sumo $3$ puntos y cualquier cónica cortará a $\mathcal{C}_F$ en a lo sumo $6$ puntos,
ya que $F$ no tiene factor común con ningún polinomio de grado $1$ y $2$ por ser
irreducible. También es claro que $F,G,H\in\mathcal{S}$. El polinomio $G$
no puede ser un múltiplo escalar de $H$, ya que si no $\mathcal{C}_G=\mathcal{C}_H$ y
su intersección sería infinita. Como $\dim(\mathbb{S})=2$, debe ser $\mathbb{S}=\langle
G,H\rangle$ y por lo tanto $F$ es combinación lineal de $G$ y $H$. Como $G(A_9)=H(A_9)=0$,
también se tiene $F(A_9)=0$.
\end{proof}

\begin{proof}[Sketch of the proof of theorem~\ref{teor-suma}]
La operación $*$ es conmutativa por definición, por lo que la operación de suma también
lo será: $P+Q=\mathcal{O}*(P*Q)=\mathcal{O}*(Q*P)=Q+P$. El elemento $\mathcal{O}$ es el
neutro aditivo, ya que $\mathcal{O}+P=\mathcal{O}*(\mathcal{O}*P)=P$ por definición.
El inverso aditivo de un $P$ es $Q=(\mathcal{O}*\mathcal{O})*P$. En efecto,
\[
   P+Q=\mathcal{O}*(P*Q)=\mathcal{O}*(P*((\mathcal{O}*\mathcal{O})*P))=
   \mathcal{O}*(\mathcal{O}*\mathcal{O})=\mathcal{O}.
\]
Solo falta demostrar la asociatividad de la suma, es decir, que $P+(Q+R)=(P+Q)+R$.
Esto es equivalente a ver que $P*(Q+R)=(P+Q)*R$. Consideremos los polinomios homogéneos
de grado $3$ dados por los siguientes productos de tres rectas:
\[
\begin{aligned}
G & = L_{\mathcal{O},P*Q,P+Q}\cdot L_{Q,R,Q*R}\cdot L_{P,Q+R,P*(Q+R)} \\
H & = L_{\mathcal{O},Q*R,Q+R}\cdot L_{P,Q,P*Q}\cdot L_{R,P+Q,(P+Q)*R}
\end{aligned}
\]
Por construcción $G$ y $H$ se anulan en los ocho puntos $\mathcal{O}$, $P$, $Q$, $R$,
$P*Q$, $P+Q$, $Q*R$, $Q+R$. Como estos $8$ están sobre la curva elíptica, el noveno
punto de intersección $\mathcal{C}_G\cap\mathcal{C}_H$ también estará sobre $\mathcal{C}_F$.
Como $F$ y $G$ se cortan en los ocho puntos mencionados antes y en $P*(Q+R)$, debe ser
que el noveno punto de intersección de $G$ y $H$ sea $P*(Q+R)$. Haciendo el razonamiento
simétrico, cambiando $G$ por $H$, concluímos que el noveno punto es $(P+Q)*R$. Esto
prueba que $P*(Q+R)=(P+Q)*R$, tal como queríamos. Un detalle técnico es que este 
razonamiento asume que los nueve puntos son distintos entre sí. Dejo como ejercicio
(difícil) completar los casos que faltan.
\end{proof}

\begin{prob}
Las curvas \emph{cusp} y \emph{node} se definen con las ecuaciones
$y^2z-x^3=0$ y $y^2z-x^3-x^2z=0$, respectivamente. Mostrar que el punto
$S=[0:0:1]$ está en ambass curvas y es su único punto singular.
\end{prob}

\begin{prob}
Calcular los puntos de intersección de la recta $x+y=0$ con las curvas cusp y
node y en cada uno de esos puntos, las multiplicidades de intersección.
\end{prob}

\begin{prob}\label{prob73}
Sea $K$ un cuerpo con ${\rm char}(K)\neq 2$. Demostrar que la curva definida por
\[F(x,y,z)=y^2z-x^3-xz^2=0\] es una curva elíptica que pasa por el punto
$\mathcal{O}=[0:1:0]$. Implementar la función \texttt{puntos(p)} que tome un primo
$p\geq 3$ y devuelva la lista de puntos de $\mathcal{C}_F(\ZZ/p\ZZ)$, cada uno de
estos escritos como una tupla de tres componentes. Tener especial cuidado
de no repetir el mismo punto dos veces en la lista (esto es no trivial, ya que un mismo
punto puede tener varias representaciones $[x:y:z]$).
\end{prob}

\begin{prob}\label{prob74}
En la curva del problema~\ref{prob73}, definimos la adición utilizando el neutro
$\mathcal{O}=[0:1:0]$. Mostrar que la tangente a la curva en $\mathcal{O}$ corta
a la curva solamente en el punto $\mathcal{O}$ con multiplicidad de intersección $3$
y por lo tanto $\mathcal{O}*\mathcal{O}=\mathcal{O}$. Mostrar que el inverso de un
punto $P=[x:y:1]$ de la curva es $-P=[x:-y:1]$. Implementar en Python3 las funciones
\texttt{inverso(P,p)} y \texttt{suma(P,Q,p)} que calculen el inverso y la suma de
puntos $P,Q\in\mathcal{C}_F(\ZZ/p\ZZ)$. ¿Cuál es la estructura de los grupos para
$p=3,5,7,11,13$? ¿Cuántos puntos de orden $2$ y $3$ hay en cada uno de esos casos?
\end{prob}

\begin{prob}
Un punto $P$ de una curva elíptica se dice de inflexión si $P*P=P$, es decir, la
tangente que pasa por $P$ corta a la curva solamente en $P$ con multiplicidad $3$.
Demostrar que en la curva del problema~\ref{prob73}, un punto $P$ es de inflexión
si y solo si $3P=\mathcal{O}$. Mostrar que en una curva elíptica cualquiera, la
condición de ser punto de inflexión es $3P=\mathcal{O}*\mathcal{O}$.
\end{prob}

\begin{prob}
Mostrar que una curva algebraica plana proyectiva $\mathcal{C}_F$ dada por un polinomio
irreducible $F\in\overline{K}[x,y,z]$ de grado $3$ no puede tener dos puntos singulares.

{\bf Hint:} Analizar las multiplicidades de intersección de la recta que pasaría por
dos puntos singulares con la curva.
\end{prob}

\begin{prob}
La curva de \emph{Fermat} está definida por la ecuación $x^3+y^3-z^3=0$. Demostrar que es
una curva elíptica si ${\rm char}(K)\neq3$. Consideremos el punto $\mathcal{O}=[1:0:1]$
de la curva y la adición definida con ese neutro. ¿Cuál es el inverso de un punto
$[x:y:z]$ de la curva? Implementar las funciones \texttt{puntos(p)}, \texttt{inverso(P,p)}
y \texttt{suma(P,Q,p)} para $\mathcal{C}_F(\ZZ/p\ZZ)$ como en los problemas~\ref{prob73}
y~\ref{prob74}.
\end{prob}

\begin{prob}
Sea $K$ un cuerpo tal que ${\rm char}(K)\neq 2,3$. Una curva en forma de Weierstrass
corresponde con la ecuación
\[
   F(x,y,z)=y^2z-x^3-axz^2-bz^3=0
\]
para ciertos $a,b\in K$. Demostrar que $\mathcal{C}_F$ es elíptica si y solo si
el polinomio $x^3+ax+b$ no tiene raíces múltiples en $\overline{K}$, o equivalentemente,
que $4a^3+27b^2\neq 0$. Mostrar que en estas curvas, el punto $\mathcal{O}=[0:1:0]$
es de inflexión. Tomando a $\mathcal{O}$ como neutro de la adición, obtener fórmulas
para el inverso y la suma. Notar que el problema~\ref{prob73} es un caso particular
de este. Implementar las funciones \texttt{puntos(a,b,p)}, \texttt{inverso(P,a,b,p)},
\texttt{suma(P,Q,a,b,p)} y \texttt{mult(k,P,a,b,p)} para trabajar con estas
curvas elípticas. La función \texttt{mult(k,P,a,b,p)} debe calcular $kP$ con el método de
la exponenciación binaria para cualquier $k\in\ZZ$.
\end{prob}

\end{document}