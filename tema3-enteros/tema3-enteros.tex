\documentclass[a4paper, 11pt]{article}

\usepackage[utf8]{inputenc}
\usepackage{amsmath,amssymb,amsthm}
\usepackage[margin=2cm]{geometry}
\usepackage{listings}
\usepackage{color}

\definecolor{deepblue}{rgb}{0,0,0.5}
\definecolor{deepred}{rgb}{0.6,0,0}
\definecolor{deepgreen}{rgb}{0,0.5,0}
\definecolor{lightgrey}{rgb}{0.95,0.95,0.95}

\renewcommand*{\lstlistingname}{Programa}

\lstset{
   language=Python,
   basicstyle=\ttfamily\small,
   keywordstyle=\color{deepblue}\bfseries\itshape,
   commentstyle=\color{deepgreen}\itshape,
   stringstyle=\color{deepred},
   backgroundcolor=\color{lightgrey},
   morekeywords={as,assert,nonlocal,with,yield},
   showstringspaces=false,
   numbers=left,
   rulecolor=\color{black},
   captionpos=b,
   frame=leftline,
   literate=
   {á}{{\'a}}1 {é}{{\'e}}1 {í}{{\'i}}1 {ó}{{\'o}}1 {ú}{{\'u}}1
   {Á}{{\'A}}1 {É}{{\'E}}1 {Í}{{\'I}}1 {Ó}{{\'O}}1 {Ú}{{\'U}}1
   {à}{{\`a}}1 {è}{{\`e}}1 {ì}{{\`i}}1 {ò}{{\`o}}1 {ù}{{\`u}}1
   {À}{{\`A}}1 {È}{{\`E}}1 {Ì}{{\`I}}1 {Ò}{{\`O}}1 {Ù}{{\`U}}1
   {ä}{{\"a}}1 {ë}{{\"e}}1 {ï}{{\"i}}1 {ö}{{\"o}}1 {ü}{{\"u}}1
   {Ä}{{\"A}}1 {Ë}{{\"E}}1 {Ï}{{\"I}}1 {Ö}{{\"O}}1 {Ü}{{\"U}}1
   {â}{{\^a}}1 {ê}{{\^e}}1 {î}{{\^i}}1 {ô}{{\^o}}1 {û}{{\^u}}1
   {Â}{{\^A}}1 {Ê}{{\^E}}1 {Î}{{\^I}}1 {Ô}{{\^O}}1 {Û}{{\^U}}1
   {ã}{{\~a}}1 {ẽ}{{\~e}}1 {ĩ}{{\~i}}1 {õ}{{\~o}}1 {ũ}{{\~u}}1
   {Ã}{{\~A}}1 {Ẽ}{{\~E}}1 {Ĩ}{{\~I}}1 {Õ}{{\~O}}1 {Ũ}{{\~U}}1
   {œ}{{\oe}}1 {Œ}{{\OE}}1 {æ}{{\ae}}1 {Æ}{{\AE}}1 {ß}{{\ss}}1
   {ű}{{\H{u}}}1 {Ű}{{\H{U}}}1 {ő}{{\H{o}}}1 {Ő}{{\H{O}}}1
   {ç}{{\c c}}1 {Ç}{{\c C}}1 {ø}{{\o}}1 {Ø}{{\O}}1 {å}{{\r a}}1 {Å}{{\r A}}1
   {€}{{\euro}}1 {£}{{\pounds}}1 {«}{{\guillemotleft}}1
   {»}{{\guillemotright}}1 {ñ}{{\~n}}1 {Ñ}{{\~N}}1 {¿}{{?`}}1 {¡}{{!`}}1 
}

\newcommand{\NN}{\mathbb{N}}
\newcommand{\ZZ}{\mathbb{Z}}
\newcommand{\QQ}{\mathbb{Q}}
\newcommand{\RR}{\mathbb{R}}
\newcommand{\CC}{\mathbb{C}}
\newcommand{\FF}{\mathbb{F}}
\newcommand{\UU}{\mathcal{U}}

\newcounter{numerodetema}
\newcommand\tema[1]{{\eject\stepcounter{numerodetema}\bf Álgebra Computacional
2023/24 \hfill Tema~\#\arabic{numerodetema}}\par\smallskip\hrule\bigskip\par
\begin{center}{\Large\bf #1}\end{center}}

\theoremstyle{plain}
\newtheorem{teor}{Teorema}[numerodetema]
\newtheorem{coro}[teor]{Corolario}
\newtheorem{lema}[teor]{Lema}
\newtheorem{prop}[teor]{Proposición}
\theoremstyle{definition}
\newtheorem{defi}[teor]{Definición}
\newtheorem{prob}{Problema}[numerodetema]

\newcommand\myatop[2]{\genfrac{}{}{0pt}{}{#1}{#2}}

\parindent=0cm

\begin{document}

\setcounter{numerodetema}{2}
\tema{Aritmética de enteros largos}

Si bien Python3 es un lenguaje de alto nivel que permite trabajar con
enteros de cualquier cantidad de dígitos, y por lo tanto no necesitamos
preocuparnos en implementar las operaciones aritméticas en términos de
los dígitos, es interesante ver como estas podrían ser implementadas si
eso no fuera así.

\bigskip

A la cantidad de operaciones elementales con dígitos (o más precisamente,
enteros cortos con los que el procesador puede trabajar) que requiere un
algoritmo se la llama \emph{complejidad binaria}. Esta noción corresponde
con el tiempo de ejecución de un algoritmo de forma independiente de la
máquina en la que el algoritmo se ejecute.

\bigskip

Por simplicidad, vamos a implementar solamente la suma, resta, multiplicación,
división (con resto) y comparación de números enteros no negativos. Utilizaremos
la siguiente representación: un número $n\geq 0$ se corresponderá con la lista
\texttt{[d\_0,d\_1,\ldots,d\_k]} donde \texttt{d\_0} es el dígito de las
unidades, $\texttt{d\_1}$ las decenas, y así siguiendo. Supondremos que 
\texttt{d\_k} es siempre no nulo. El número $0$ se representará con la lista
vacía \texttt{[]}.

\bigskip

A continuación vamos a ver como implementar los algoritmos de la escuela para
la suma, resta, comparación y multiplicación de números naturales. Vamos a
poner todas la funciones en el fichero \texttt{natural.py} para luego
poder utilizarlas en otros programas.

\bigskip

La función \texttt{remover\_ceros(a)} toma una lista $a$ de
dígitos y le elimina los ceros a la izquierda, es decir, la reduce a una
representación válida de acuerdo con nuestra convención.

\bigskip

\lstinputlisting[language=Python,linerange={1-5},firstnumber=1,
caption={\texttt{natural.py} (lineas 1--5)},label={prog11}]
{natural.py}

\bigskip

El algoritmo de la suma se implementa dígito a dígito, empezando por
las unidades.

\bigskip

\lstinputlisting[language=Python,linerange={7-28},firstnumber=7,
caption={\texttt{natural.py} (lineas 7--28)},label={prog12}]
{natural.py}

\bigskip

De forma similar, es fácil implementar la comparación
\texttt{comparar(a,b)} que devuelve $1,0,-1$ dependiendo
si $a>b$, $a=b$ o $a<b$, y la resta \texttt{restar(a,b)}
que calcula $a-b$, suponiendo que $a\geq b$. La resta se hace empezando
por las unidades, pero la comparación empieza por el dígito más
significativo. Cuando la resta es invocada con $a<b$, la función
no devuelve nada.

\bigskip

\lstinputlisting[language=Python,linerange={30-47},firstnumber=30,
caption={\texttt{natural.py} (lineas 30--47)},label={prog13}]{natural.py}

\bigskip

\lstinputlisting[language=Python,linerange={49-74},firstnumber=49,
caption={\texttt{natural.py} (lineas 49--74)},label={prog14}]{natural.py}

\bigskip

Las funciones \texttt{sumar(a,b)}, \texttt{comparar(a,b)} y \texttt{restar(a,b)}
reciben una entrada de tamaño $n=len(a)+len(b)$ y se puede comprobar que sólo 
requieren $O(n)$ operaciones entre dígitos.

\bigskip

La multiplicación mediante el algoritmo de la escuela es también fácil de
implementar, pero luego veremos que no es el método más apropiado para números
muy largos.

\bigskip

\lstinputlisting[language=Python,linerange={76-88},firstnumber=76,
caption={\texttt{natural.py} (lineas 76--88)},label={prog15}]{natural.py}

\bigskip

Aquí la cantidad de operaciones es del orden de $len(a)\cdot len(b)$, es
decir, \texttt{multiplicar(a,b)} tiene complejidad binaria $O(n^2)$,
donde $n=len(a)+len(b)$ es el tamaño de la entrada. El peor caso
sucede cuando $len(a)=len(b)=n/2$.

\bigskip

Es posible multiplicar números naturales de forma mucho más eficiente.
El algoritmo de Karatsuba consiste en reducir un producto de números de
$n$ dígitos a sólo 3 productos de números de $d\approx n/2$ dígitos.
La clave está en observar que el producto
\[
   (a_110^d + a_0)\cdot (b_110^d + b_0) =
   (c_210^{2d} + c_110^d + c_0)
\]
se puede hacer con sólo 3 multiplicaciones:
\[
\begin{aligned}
   c_2  &= a_1\cdot b_1, \\
   c_0  &= a_0\cdot b_0, \\
   c_1  &= (a_1+a_0)\cdot (b_1+b_0)-c_2-c_0.
\end{aligned}
\]
Cuando los números son suficientemente pequeños (caso base), se multiplican
directamente utilizando el algoritmo de la escuela.

\bigskip

La complejidad aritmética $M(n)$ del algoritmo de Karatsuba se puede determinar
resolviendo la siguiente recursión:
\[
   M(n)\leq 3 M(\lceil n/2\rceil) + Cn
\]
donde $C>0$ es una cierta constante que representa las sumas y restas que se
necesitan para calcular $c_1$ y para conseguir el resultado final a partir de
$c_0$, $c_1$ y $c_2$.

\bigskip

Para un $n=2^r$, se puede ver que
\[
\begin{aligned}
   M(n)=M(2^r) &\leq 3\cdot M(2^{r-1})+C\cdot 2^r \\
          &\leq 3^2\cdot M(2^{r-2})+C\cdot 3\cdot 2^{r-1}+C\cdot 2^r \\
          &\qquad\vdots \\
          &\leq 3^r\cdot M(1)+C\cdot
            (3^{r-1}\cdot 2+3^{r-2}\cdot 2^2+\cdots+3\cdot 2^{r-1}+2^r) \\
          &=3^r\cdot M(1) + C\cdot 3^{r-1}\cdot 2\cdot\frac{1-(2/3)^r}{1-(2/3)} \\
          &\leq 3^r\cdot (M(1) + 2\cdot C) \\
          &=O(n^{\log_23}).
\end{aligned}
\]
Para cualquier $n$, se puede siempre elegir un $n\leq 2^r<2n$ y luego aplicar el
razonamiento anterior. Se prueba así que $M(n)=O(n^{\log_23})=O(n^{1.585})$ lo
que es mucho más eficiente (para $n$ grande) que la multiplicación de la escuela
cuya complejidad binaria es $O(n^2)$.

\bigskip

\lstinputlisting[language=Python,linerange={90-121},firstnumber=90,
caption={\texttt{natural.py} (lineas 90--121)},label={prog16}]
{natural.py}

\bigskip

\begin{prob}\label{prob-b10-b2}
Implementar la función \texttt{div2(a)} que calcule los dígitos de
$\lfloor a/2\rfloor$, es decir, que se comporte como la operación $a//2$ de
Python3 pero utilizando la representación de listas de dígitos ``reducidas''.
Utilizar esa función para implementar la función \texttt{decimal\_a\_base2(a)}
que devuelve la representación binaria de $a$ (una lista de ceros y unos).
Calcular la complejidad binaria de ambas funciones.
\end{prob}

\begin{prob}
Implementar la función \texttt{base2\_a\_decimal(b)} que obtenga la lista de
dígitos decimales de un número a partir de la lista de sus bits. Calcular la
complejidad binaria de esta función.
\end{prob}

\begin{prob}\label{prob-complexity}
Sea $f:\NN\to\RR_{>0}$ una función creciente tal que
\[
  f(n)\leq k\cdot f(\lceil n/r\rceil)+C\cdot n^\alpha\quad \forall\,n\in\NN
\]
para ciertos $C,\alpha\geq 0$, $k>1$ y $r\geq 2$ entero. Demostrar que:
\begin{itemize}
\item $\alpha < \log_rk\;\Longrightarrow\; f\in O(n^{\log_rk})$,
\item $\alpha = \log_rk\;\Longrightarrow\; f\in O(n^\alpha\log n)$,
\item $\alpha > \log_rk\;\Longrightarrow\; f\in O(n^\alpha)$.
\end{itemize}
\end{prob}

\begin{prob}
Demostrar que es posible multiplicar dos matrices de $n\times n$ con
$O(n^{\log_27})$ operaciones aritméticas. ¿Cuál es la cantidad de
operaciones aritméticas si se las multiplica directamente haciendo los
productos de filas con columnas?
\end{prob}

\begin{prob}
Hacer una gráfica que muestre los verdaderos tiempos de ejecución de la
función \texttt{multiplicar\_karatsuba(a,b)} (medidos en segundos) en función
de la cantidad de dígitos totales de entrada $n=len(a)+len(b)$ y comparar con
$n^{log_23}$.
\end{prob}

\begin{prob}\label{prob-toom-3}
El algoritmo Toom-3 para multiplicar números naturales es una extensión
del de Karatsuba dividiendo a cada factor en tres partes. Más precisamente,
\[
   (a_210^{2d}+a_110^d+a_0)\cdot(b_210^{2d}+b_110^d+b_0) =
   (c_410^{4d}+c_310^{3d}+c_210^{2d}+c_110^d+c_0),
\]
donde $c_0,\ldots,c_4$ pueden calcularse del siguiente modo:
\[
\begin{aligned}
   c_0 &= a_0\cdot b_0, \\
   c_4 &= a_2\cdot b_2, \\
   s_1 &= (a_0+a_1+a_2)\cdot(b_0+b_1+b_2),     \\
   s_2 &= (a_0+2a_1+4a_2)\cdot(b_0+2b_1+4b_2), \\
   s_3 &= (a_0+3a_1+9a_2)\cdot(b_0+3b_1+9b_2), \\
   c_1 &= -\frac{11}6c_0+3s_1-\frac32s_2+\frac13s_3-6c_4, \\
   c_2 &= c_0-\frac52s_1+2s_2-\frac12s_3+11c_4, \\
   c_3 &= -\frac16c_0+\frac12s_1-\frac12s_2+\frac16s_3-6c_4.
\end{aligned}
\]
Demostrar que el método es correcto e implementarlo (se necesitará implementar
también la función \texttt{div6(a)} que equivale a la operación \texttt{a//6}
de Python3). ¿Cuál es la complejidad binaria de este algoritmo?
\end{prob}

\begin{prob}\label{prob_gcd_2}
Reimplementar el algoritmo \texttt{gcd\_binario(x,y)} del programa \texttt{gcds.py}
del tema~\#2 para $x,y\geq 0$ representados por listas de dígitos (reducidas) y
demostrar que su complejidad es $O(n^2)$ donde $n=len(x)+len(y)$. Parte del ejercicio
consiste en implementar la función~\texttt{mul2(x)} que calcule el doble de
un entero $x\geq 0$.
\end{prob}

\begin{prob}
Implementar la función~\texttt{raiz\_cuadrada(x)} que dado un entero (no
negativo) calcule su raíz cuadrada $\lfloor\sqrt{x}\rfloor$, utilizando
una búsqueda binaria o bien obteniendo dígito por dígito del resultado.
Tanto la entrada como la salida del algoritmo serán listas de dígitos.
¿Cuál es la complejidad binaria del método? Generalizar a la
función~\texttt{raiz\_k\_esima(x,k)} que calcula $\lfloor\sqrt[k]{x}\rfloor$,
donde $x$ es una lista de dígitos y $k\geq 1$ un entero.
\end{prob}

\bigskip

Hasta ahora hemos visto que la suma, resta y comparación de números
naturales de $n$ dígitos puede hacerse en $O(n)$ operaciones elementales
entre dígitos. Si se usa el método de la escuela, la multiplicación puede
hacerse en $O(n^2)$ operaciones, y si se usa el método de Karatsuba, la
cantidad de operaciones se reduce a $O(n^{1.585})$.

\bigskip

El algoritmo de la escuela para dividir enteros largos suele ser presentado
informalmente. Para dividir un número $a$ de $n$ dígitos por uno $b$ de
$m\leq n$ dígitos, se busca el mayor entero $0\leq c\leq 9$ tal que
$r=a-b\cdot c\cdot 10^{n-m}\geq 0$. Luego se agrega $c\cdot 10^{n-m}$ al
cociente y se reemplaza $a$ por $r$. Ese procedimiento reduce la cantidad
de dígitos de $a$, o en el caso de ser $c=0$ nos muestra que $a<b\cdot 10^{n-m}$.
En este último caso, y suponiendo que $n>m$, se busca el mayor dígito $c$ tal que
$r=a-b\cdot c\cdot 10^{n-m-1}\geq 0$ y se añade $c\cdot 10^{n-m-1}$ al cociente.
El método termina cuando se llega a $a<b$.

\bigskip

\lstinputlisting[language=Python,linerange={123-148},firstnumber=123,
caption={\texttt{natural.py} (lineas 123--148)},label={prog17}]{natural.py}

\bigskip

En Python3, la expresión $a[i:j]$ es la lista formada por
$a[i],a[i+1],\ldots,a[j-1]$. En caso de no especificarse $i$, se
entiende que $i=0$, es decir, $a[:j]$ es la lista de las primeras $j$ entradas
de $a$. Del mismo modo, si no se especifica $j$, se entiende que $j=len(a)$,
es decir, $a[i:]$ es la lista de todas las entradas de $a$ a partir de $a[i]$.
Esto mismo se puede usar con tuplas.

\bigskip

Es fácil comprobar que \texttt{division\_y\_resto(a,b)} require una cantidad
$O(n^2)$ operaciones, donde $n=len(a)+len(b)$ es el tamaño de la entrada. El
peor caso sucede cuando $len(b)\approx len(a)/2$.

\bigskip

Parece bastante ineficiente tener que probar con todos los posibles
dígitos $c=9,8,\ldots,0$ hasta dar con el correcto en las lineas 8--10 y
15--17. Si bien la cantidad máxima de intentos es 10, eso puede empeorar
significativamente si se trabaja con dígitos en una base mas grande.

\bigskip

Suponiendo que el dígito más significativo de $b$ satisface $b_{m-1}\geq 5$,
se puede ver que el resultado de las lineas 8--10 será siempre $c=0$ o $c=1$
dependiendo de si $a<b\cdot 10^{n-m}$ o no. De forma similar, se puede
ver que el resultado de las lineas 15--17 será
\[
   c^*=\min\left\{
     \left\lfloor\frac{a_{n-1}\cdot 10+a_{n-2}}{b_{m-1}}\right\rfloor,9
     \right\}
\]
o $c^*-1$ o $c^*-2$. Esto requiere a lo sumo tres intentos hasta
acertar con el dígito correcto. Además, este método funciona bien en
cualquier base $B$, suponiendo que $b_{m-1}\geq\lfloor B/2\rfloor$.
A esto se lo llama ``división normalizada''.

\bigskip

Para conseguir llegar siempre al caso normalizado, se puede premultiplicar
a $a$ y $b$ por el dígito $f=\lfloor B/(b_{m-1}+1)\rfloor$ antes de hacer
la división. El cociente será el mismo y el resto saldrá multiplicado por $f$,
por lo que al final habrá que dividirlo por $f$. Por supuesto, para esto
habrá que implementar cuidadosamente un algoritmo para dividir por un
número de un sólo dígito.

\begin{prob}
Incorporar las mejoras indicadas arriba a la función
\texttt{division\_y\_resto(a,b)}. Comparar los tiempos reales de ejecución de
ambas versiones.
\end{prob}

Es de esperar que el algoritmo de la escuela para hacer divisiones no sea
lo más eficiente. A continuación vamos a mostrar que se puede conseguir
dividir en $D(n)=O(M(n))$ operaciones, es decir, que la complejidad binaria
de la multiplicación y la de la división con resto son del mismo orden. El
truco será ver que es posible dividir haciendo unas pocas multiplicaciones.

\bigskip

Digamos que $a$ es de $n$ dígitos y $b$ de $m$ dígitos. Para calcular
$\lfloor a/b\rfloor$ podemos primero calcular la expansión decimal de $1/b$
con $n$ dígitos de precisión (es decir, un error absoluto menor que $10^{-n}$),
luego multiplicar esos $n$ dígitos por $a$ y por último tomar la parte
entera. Más precisamente, vamos a calcular un aproximación entera $y$ del
número real $10^n/b$ con error absoluto menor que $1$ y luego hacer
\[
   \left\lfloor\frac{a}{b}\right\rfloor=
   \left\lfloor
       \frac{a\cdot y}
            {10^{n}}
   \right\rfloor+\left\{
     \begin{array}{c} 0 \\ \pm1 \end{array}
   \right\}
\]
La división entera por $10^n$ es fácil de hacer, ya que estamos trabajando
con la representación en base~$10$.

\bigskip

Para calcular la aproximación buscada de $1/b$ vamos a utilizar el 
método de Netwon.

\begin{teor}
Sea $f:(r-\varepsilon,r+\varepsilon)\subseteq\RR\to\RR$ una función
analítica en $r$ con radio de convergencia $\varepsilon>0$, es decir,
\[
   f(x)=\sum_{i=0}^\infty\frac{f^{(i)}(r)}{i!}(x-r)^i
\]
para todo $x\in (r-\varepsilon,r+\varepsilon)$. Supongamos que $f(r)=0$
y que $f'(r)\neq 0$. Sea
\[
  \gamma=\sup_{i\geq 2}\left|\frac{f^{(i)}(r)}{i!f'(r)}\right|^{1/(i-1)}.
\]
Si $|x-r|<\frac{1}{5\gamma}$, entonces $f'(x)\neq 0$ y
$|x-f(x)/f'(x)-r|\leq 5\gamma|x-r|^2<\frac{1}{5\gamma}$. En particular, si
hacemos la iteración $x_{k+1}=x_{k}-f(x_k)/f'(x_k)$ partiendo de un $x_0$ tal
que $|x_0-r|<\frac{1}{50\gamma}$, tendremos que
$|x_k-r|<\frac{1}{5\gamma}10^{-2^k}$ para todo $k\geq 0$.
\end{teor}
\begin{proof}
Como $f$ es analítica, $f'$ también lo es (en el mismo intervalo) y su
desarrollo en series centrado en $r$ es:
\[
   f'(x)=\sum_{i=1}^\infty\frac{f^{(i)}(r)}{(i-1)!}(x-r)^{i-1}.
\]
Suponiendo que $|x-r|<\frac1{5\gamma}$, tenemos que
\[
\begin{aligned}
  |f'(x)| &\geq |f'(r)|\left[
     1-\sum_{i=2}^\infty\left|\frac{f^{(i)}(r)}{(i-1)!f'(r)}\right|
     \left(\frac1{5\gamma}\right)^{i-1}
  \right] \\
  &\geq |f'(r)|\left[
     1-\sum_{i=2}^\infty i\gamma^{i-1}\left(\frac1{5\gamma}\right)^{i-1}
  \right] = \frac7{16}|f'(r)|
\end{aligned}
\]
y en particular, $f(x)\neq 0$. Además, tenemos que
\[
\begin{aligned}
\left|x-\frac{f(x)}{f'(x)}-r\right| &= \frac{|(x-r)f'(x)-f(x)|}{|f'(x)|}
\leq \frac{16}{7|f'(r)|}\left|\sum_{i=2}^\infty \frac{(i-1)f^{(i)}(r)}{i!}(x-r)^i\right| \\
&\leq \frac{16}7|x-r|^2\sum_{i=2}^\infty(i-1)\left|\frac{f^{(i)}(r)}{i!f'(r)}\right||x-r|^{i-2} \\
&\leq\frac{16}7|x-r|^2\sum_{i=2}^\infty(i-1)\gamma^{i-1}\left(\frac1{5\gamma}\right)^{i-2}=\frac{25}7\gamma|x-r|^2 \\
& <5\gamma|x-r|^2<\frac1{5\gamma}
\end{aligned}
\]
con lo que queda probada la primera parte. Lo de la velocidad de convergencia
de la iteración de Newton es hacer inducción utilizando la desigualdad que
acabamos de probar.
\end{proof}

En nuestro caso, vamos a utilizar la función $f(x)=\frac{1}{x}-b$ cuyo cero
está en el punto $r=1/b$. Claramente $f(x)$ es analítica en $r=1/b$  y un
cálculo simple muestra que $\gamma=b$. La iteración del método de Netwon
es $x_{k+1}=x_k-f(x_k)/f'(x_k)=x_k(2-bx_k)$ que, afortunadamente, sólo
requiere una resta y dos multiplicaciones, pero ninguna división.

\bigskip

Debemos buscar, para empezar, un $x_0$ tal que
$\left|x_0-\frac1b\right|<\frac1{50b}$. Suponiendo que $10^{m-1}\leq b<10^m$,
es decir, que $b$ tiene $m$ dígitos, resulta que $r=1/b=0.0\cdots0\ast\ast\cdots$,
donde hay exactamente $m-1$ ceros después del punto decimal. Un $x_0$ que tenga
los primeros tres dígitos no nulos correctos de $1/b$ satisface
$\left|x_0-\frac1b\right|<10^{-m-2}<\frac1{50b}$ y por lo tanto nos sirve
como punto de partida. Si pensamos a $x_0=y_0\cdot 10^{-m-2}$ con $y_0\in\NN$
de tres dígitos, lo que necesitamos es que $|y_0-10^{m+2}/b|<1$, es decir,
\[
  y_0=\left\lfloor\frac{10^{m+2}}b\right\rfloor=\max\left\{t\in\NN\;:\;100<
  t\leq 1000,\;bt\leq 10^{m+2}\right\}.
\]
El valor de $y_0$ puede determinarse explorando todos los posibles $101\leq t\leq
1000$, o bien, haciendo una búsqueda binaria. En cualquier caso, la cantidad de
operaciones que se necesitan es $O(m)$.

\bigskip

La iteración de Newton nos garantiza que $|x_k-1/b|<\frac1{5b}10^{-2^k}$, por
lo que $x_k$ tendrá al menos $2^k$ dígitos correctos (comparados con $1/b$)
luego de los $m-1$ ceros que vienen a partir del punto decimal. Por otra parte,
necesitamos garantizar que $2+2^k$ dígitos de $x_k$ estén almacenados correctamente
en memoria para que $x_{k+1}$ tenga la precisión indicada. De todo esto,
concluímos que $x_k$ será de la forma $y_k10^{-m-1-2^k}$ con $y_k$ un número
natural de $2+2^k$ dígitos decimales.
\[
   y_{k+1}10^{-m-1-2^{k+1}}\approx y_k10^{-m-1-2^k}(2-by_k10^{-m-1-2^k})
\]
\[
   y_{k+1}\approx y_k10^{2^k}(2-by_k10^{-m-1-2^k})
\]
\[
  y_{k+1}\approx 2y_k10^{2^k}-by_k^210^{-m-1}
\]
\[
  y_{k+1}=2y_k10^{2^k}-\left\lfloor\frac{\left\lfloor\frac{b}{10^{m-3-2^{k+1}}}
  \right\rfloor y_k^2}{10^{2^{k+1}+4}}\right\rfloor
\]
Es fácil ver que la iteración requiere $O(M(2^k))$ operaciones binarias para
$k=0,1,\ldots,\lfloor\log_2(n)\rfloor$. En la iteración final, obtenemos la
aproximación de $1/b$ con $n$ dígitos que buscabamos. La complejidad total
del método queda $O(M(n))$.

\bigskip

Veamos un ejemplo concreto. Supongamos que $b=56273694826793487298234$ de $m=23$
dígitos decimales.
\[
\begin{aligned}
   x_0 &=y_0\cdot 10^{-25}\qquad
   y_0=\max\left\{t\in\NN\;:\;101\leq t\leq 1000,\;bt\leq 10^{25}\right\}=177 \\
   x_1 &=y_1\cdot 10^{-26}\qquad
   y_1=2\cdot177\cdot10^1-
   \left\lfloor\frac{\lfloor b/10^{18}\rfloor 177^2}{10^6}\right\rfloor=1778 \\
   x_2 &=y_2\cdot 10^{-28}\qquad
   y_2=2\cdot1778\cdot10^2-
   \left\lfloor\frac{\lfloor b/10^{16}\rfloor 1778^2}{10^8}\right\rfloor=177703 \\
   x_3 &=y_3\cdot 10^{-32}\qquad
   y_3=2\cdot177703\cdot10^4-
   \left\lfloor
     \frac{\lfloor b/10^{12}\rfloor 177703^2}{10^{12}}
   \right\rfloor=1777029220 \\
   x_4 &=y_4\cdot 10^{-40}\qquad
   y_4=2\cdot1777029220\cdot10^8-
   \left\lfloor
     \frac{\lfloor b/10^{4}\rfloor 1777029220^2}{10^{20}}
   \right\rfloor= 177702921956329746\\
   x_5 &=y_5\cdot 10^{-56}\qquad
   y_5=2\cdot177702921956329746\cdot10^{16}-
   \left\lfloor
     \frac{\lfloor b\cdot 10^{12}\rfloor 177702921956329746^2}{10^{36}}
   \right\rfloor= \\
   & \hspace{3.2cm} =1777029219563297453449602028559613 \\
   x_6 &=y_6\cdot 10^{-88}\qquad
   y_6=177702921956329745344960202855961272620268353714009725013100908268   
\end{aligned}
\]
Sabemos que la diferencia entre $1/b$ y $x_6$ es menor que $10^{-86}$, por lo
que si quisieramos dividir un número
\[
a=2934872934729487239488472984749283479283749238427947294923847293847298482014
\]
de $n=76\leq86$ cifras por $b$, bastaría con calcular
\[
q=\left\lfloor\frac{a\cdot y_6}{10^{88}}\right\rfloor=
52153549607197851357899473386105573255063608067175345
\]
para conseguir el cociente con error de a lo sumo una unidad. Se comprueba que
$bq\leq a<b(q+1)$, por lo que no hay que hacer ninguna corrección al valor de
$q$. Por último, el resto se puede calcular haciendo
$a-bq=31237543472563311641284$.

\bigskip

\begin{prob}
Extender la librería \texttt{natural.py} con una implementación del algoritmo
de la división larga utilizando la iteración de Newton.
\end{prob}

\begin{prob}
Demostrar que es posible convertir de base 10 a base 2 y viceversa con
$O(M(n))$ operaciones binarias. La idea es calcular primero las potencias
$2^{2^i}$ para $i=0,1,\ldots$ en base $10$, luego hacer una división con
resto por la mayor de esas potencias que sea menor que el número dado,
es decir, escribirlo de la forma $a+b2^{2^j}$, y finalmente convertir
tanto el cociente $b$ como el resto $a$ a binario.
\end{prob}

\begin{prob}
Calcular los primeros mil dígitos de $\sqrt{2}$ después del punto decimal,
utilizando la iteración de Newton con la función $f(x)=x^2-2$.
\end{prob}

% Falta escribir la demostracion completa del algoritmo de division.
% Explicar mejor.

\end{document}